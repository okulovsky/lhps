\documentclass[aspectratio=169]{beamer}
\usepackage{framed}
\usepackage{lphs}

\SetupImages{L2-Images/}{jpg}
\begin{document}

\section{Причинно-следственное мышление}

\begin{frame}
Необходимые и достаточные условия
\end{frame}

\section{Ошибки причинно-следственного мышления}

\begin{frame}
После не значит вследствие
\end{frame}

\begin{frame}
Корреляция не означает причинности
\end{frame}

Глава заканчивается почти анекдотическим (но реальным) примером перепутывания причины и следствия аборигенами Новых Гебрид. Они полагали, что наличие вшей ведёт к здоровью. Этот вывод делался на том основании, что больного человека вши покидали (так как вследствие повышенной температуры тела условия существования для них становились некомфортными), тогда как у всех здоровых людей они были (иными словами, наблюдалась положительная корреляция между здоровьем и наличием вшей).

\section{Абдукция}

\section{Аналогия}

\end{document}