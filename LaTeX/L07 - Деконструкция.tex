\documentclass[aspectratio=169]{beamer}
\usepackage{framed}
\usepackage{lphs}

\SetupImages{L2-Images/}{jpg}
\begin{document}

\section{Что такое деконструкция?}

\begin{Person}{deMan}{Поль де Ман}{1919 -- 1983}
\Citation{
Из опыта чтения абстрактных философских текстов нам всем знакомо облегчение, которое мы чувствуем, когда рассуждение сменяется тем, что называется <<конкретным примером>>. Однако именно в этот самый момент, когда нам наконец кажется, что мы все поняли --- мы, в действительности, дальше от понимания, чем когда-либо прежде}{}
\end{Person}

\begin{frame}
\begin{center}
\begin{tikzpicture}[x=3cm,y=-0.9cm]
\uncover<+->{\node at (-1,-1) {\includegraphics[angle=10, width=7cm]{L2-Images/nose1.jpg}};}
\uncover<+->{\node at (1,-1) {\includegraphics[angle=-5, width=7cm]{L2-Images/nose3.jpg}};}
\uncover<+->{\node at (1,1) {\includegraphics[angle=3, width=7cm]{L2-Images/nose2.jpg}};}
\uncover<+->{\node at (-1,1) {\includegraphics[angle=-2, width=7cm]{L2-Images/nose4.jpg}};}
\uncover<+->{\node at (0,0) {\includegraphics[width=10cm]{L2-Images/nose5.jpg}};}
\end{tikzpicture}
\end{center}
\end{frame}


\begin{frame}
\begin{center}
\begin{huge}
Нельзя публично ковырять в носу!
\end{huge}
\end{center}
\end{frame}

\end{document}