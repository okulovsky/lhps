\documentclass[aspectratio=169]{beamer}
\usepackage{lphs}

\begin{document}

\section{Почему важна философия познания?}

\PictureFrame{WorldEconomicHistory}{Среднедушевой доход по времени с X в. до н.э. по настоящее время в относительных единицах}

\begin{GalleryFrame}
\image{Automobile}{Автомобиль (1769)}
\image{SteamShip}{Пароход (1783)}
\image{SteamTrain}{Паровоз (1804)}
\end{GalleryFrame}

\begin{GalleryFrame}
\image{Aerostat}{Монгольфьер (1783)}
\image{SkyLantern}{Китайский летающий фонарик (III в. до н.э.)}
\image{Aircraft}{Самолет братьев Райт (19001)}
\end{GalleryFrame}

\begin{GalleryFrame}
\image{CameraObscura}{Фотография (1826)}
\image{Cinema}{Киноаппарат (1891)}
\end{GalleryFrame}

\begin{GalleryFrame}
\image{Telegraph}{Телеграф (1832)}
\image{Telephone}{Телефон (1860)}
\image{Radio}{Радио (1895)}
\end{GalleryFrame}
	
\begin{GalleryFrame}
\image{ElectricalEngine}{Электродвигатель (1834)}
\image{LightBulb}{Лампа накаливания (1840)}
\end{GalleryFrame}

\begin{GalleryFrame}
\image{Eraser}{Ластик (1738)}
\image{Mackintosh}{Макинтош (1823)}
\image{Rubber}{Вулканизированная резина (1826)}
\end{GalleryFrame}

\begin{GalleryFrame}
\image{Pilewood}{Фанера (1797)}
\image{Jeans}{Джинсы (1853)}
\image{CornFlakes}{Кукурузные хлопья (1894)}
\end{GalleryFrame}

\begin{GalleryFrame}
\image{BookPrinting}{Книгопечатный станок}
\image{Clocks}{Механические часы}
\image{Glasses}{Очки}
\end{GalleryFrame}


\section{Познание в первобытном мире}

\begin{Person}{Nongqawuse}{Нонкгавусе}{c. 1840 -- 1898}
Коса должны уничтожить их посевы и убить свой скот; взамен духи сметут британцев в море, и тогда зернохранилища и хлева коса станут полнее, чем прежде
\end{Person}

\section{Античность}

\begin{Person}{Thales}{Фалеc}{c. 624-- c. 546 BC}
Все возникает из воды и в неё превращается
\end{Person}

\begin{Person}{Anaximander}{Анаксимандр}{c. 640 -- c. 546 BC}

Земля парит в центре мира, ни на что не опираясь. Землю окружают исполинские трубчатые кольца-торы, наполненные огнём. В самом близком кольце, где огня немного, имеются небольшие отверстия — звёзды. Во втором кольце с более сильным огнём находится одно большое отверстие — Луна. В третьем, дальнем кольце, имеется самое большое отверстие, размером с Землю; сквозь него сияет самый сильный огонь — Солнце.
\end{Person}

\begin{Person}{Unknown}{Эвбулид}{IV BC}
То, что не потерял, то имеешь. Рогов ты не терял. Следовательно, у тебя есть рога
\end{Person}

\begin{Person}{Aristotle}{Аристотель}{384--322 BC}
\begin{Reason}
\from{Все люди смертны}
\from{Сократ -- человек}
\conc{Сократ -- смертен}
\end{Reason}
\end{Person}

\PictureFrame{Aerolipile}{Паровая машина Герона Александрийского, I AD}

\end{document}