\documentclass[aspectratio=169]{beamer}
\usepackage{framed}
\usepackage{lphs}

\newcommand{\task}{
В каждой комнате или принцесса, или тигр.

Если в первой комнате принцесса, то табличка на ней истинна, если тигр -- ложна. Во второй комнате наоборот.

\begin{columns}
	\column{2.5cm}
	\begin{framed}
		\begin{center}
			I
			
			Что выбрать -- большая разница!
		\end{center}
	\end{framed}
	\column{2.5cm}
	\begin{framed}
		\begin{center}
			II
			
			Лучше выбрать другую комнату.
		\end{center}
	\end{framed}
\end{columns}	
}


\begin{document}

\SetupImages{L2-Images/}{jpg}
\newcommand{\firstcircle}{(-1,0) circle (1.5)}
\newcommand{\secondcircle}{(1,0) circle (1.5)}
\newcommand{\thirdcircle}{(0,1.73) circle (1.5)}

\section{Причинно-следственная логика}

\begin{frame}\frametitle{Правила вывода}
\begin{center}
\begin{minipage}{0.5\textwidth}
\begin{Reason}
	\from{Если идет дождь, земля мокрая}
	\from{Сейчас идет дождь}
	\conc{Сейчас земля мокрая}
\end{Reason}\\[1cm]


\begin{Reason}
	\from{Если идет дождь, земля мокрая}
	\from{Земля не мокрая}
	\conc{Дождь не идет}
\end{Reason}\\[1cm]

	
\begin{Reason}
	\from{Если идет дождь, земля мокрая}
	\from{Дождь не идет}
	\conc{Земля не мокрая}
\end{Reason}
\end{minipage}
\end{center}
\end{frame}

\begin{frame}
\begin{columns}
\column{0.5\textwidth}
\task


\column{0.5\textwidth}
Если в первой комнате тигр, то утверждение I ложно. Тогда во второй тоже тигр. Тогда утверждение II истинно. Приходим к противоречию.

Если в первой комнате принцессе, то утверждение I истинно. Тогда во второй комнате тигр. Тогда утверждение II истинно. Противоречий нет.

\end{columns}
\end{frame}


\begin{frame}
\begin{columns}
\column{0.5\textwidth}

\task

\column{0.5\textwidth}

\begin{tabular}{l| l}
\uncover<2->{В первой комнате принцесса & $P_1$ \\ \hline 
}\uncover<3->{В первой комнате тигр & $\neg P_1$ \\ \hline 
}\uncover<4->{Во второй комнате принцесса & $P_2$ \\ \hline 
}\uncover<5->{Во второй комнате тигр & $\neg P_2$ \\ \hline 
}\uncover<6->{Если $A$, то $B$ & $A\rightarrow B$ \\ \hline 
}\uncover<7->{Верно и $A$, и $B$ & $A\wedge B$ \\ \hline
}\uncover<8->{Верно $A$, или $B$, или оба & $A\vee B$ \\ \hline
}\end{tabular}

$$
\uncover<10->{P_1\rightarrow}\uncover<11->{(P_1\wedge \neg P_2)\vee (\neg P_1 \wedge P_2)}
$$ $$
\uncover<12->{\neg P_1\rightarrow}\uncover<13->{(P_1\wedge P_2)\vee (\neg P_1 \wedge \neg P_2)}
$$ $$
\uncover<14->{\neg P_2\leftrightarrow}\uncover<15->{P_1 \wedge \neg P_2}
$$ 

\end{columns}
\end{frame}


\begin{frame}
\begin{columns}
\column{0.5\textwidth}


\begin{tabular}{l| l}
В первой комнате принцесса & $P_1$ \\ \hline 
В первой комнате тигр & $\neg P_1$ \\ \hline 
Во второй комнате принцесса & $P_2$ \\ \hline 
Во второй комнате тигр & $\neg P_2$ \\ \hline 
Если $A$, то $B$ & $A\rightarrow B$ \\ \hline 
Верно и $A$, и $B$ & $A\wedge B$ \\ \hline
Верно $A$, или $B$, или оба & $A\vee B$ \\ 
\end{tabular}

$$
P_1\rightarrow(P_1\wedge \neg P_2)\vee (\neg P_1 \wedge P_2)
$$ $$
\neg P_1\rightarrow(P_1\wedge P_2)\vee (\neg P_1 \wedge \neg P_2)
$$ $$
\neg P_2\leftrightarrow P_1 \wedge \neg P_2
$$ 

\column{0.5\textwidth}

$$
\uncover<2->{A,\ A\rightarrow B\ \Rightarrow\ B}
$$ $$
\uncover<3->{A\wedge B\ \Rightarrow\ A} \uncover<4->{,\ \ A\wedge B\ \Rightarrow\ B}
$$ \only<4-14>{$$
\uncover<4->{\neg P_1 \Rightarrow}\uncover<6->{(P_1\wedge P_2)\vee (\neg P_1 \wedge \neg P_2)}
$$ $$
\uncover<7->{\neg P_1, P_1\wedge P_2\Rightarrow}\uncover<8->{P_1\Rightarrow}\uncover<9->{\times}
$$ $$
\uncover<10->{\neg P_1, \neg P_1\wedge \neg P_2\Rightarrow}\uncover<11->{\neg P_2\Rightarrow}
$$ $$
\uncover<12->{\Rightarrow P_1 \wedge \neg P_2\Rightarrow }\uncover<13->{P_1 \Rightarrow}\uncover<14->{\times}
$$}\only<15->{$$
\uncover<15->{\neg P_1\Rightarrow \times}
$$ $$
\uncover<16->{P_1 \Rightarrow}\uncover<17->{(P_1\wedge \neg P_2)\vee (\neg P_1 \wedge P_2) \Rightarrow} $$ $$ 
\uncover<18->{\Rightarrow P_1\wedge \neg P_2 \Rightarrow}\uncover<19->{\neg P_2\Rightarrow}\uncover<20->{P_1\wedge\neg P_2}
$$}
\end{columns}
\end{frame}



\end{document}