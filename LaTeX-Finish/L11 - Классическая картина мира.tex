\documentclass[aspectratio=169]{beamer}
\usepackage{framed}
\usepackage{lhps}

\renewcommand{\ImageSource}{L3-Images/}
\begin{document}

\begin{Person}{Ptolemy}{Клавдий Птолемей}{ок. 90 -- ок. 168}
\Citation{
Земля находится в центре.

Солнце и Луна вращаются вокруг Земли по круговым орбитам.

Планеты вращаются вокруг Земли по эпициклам.
}{по <<Альмагест>>, 140}
\end{Person}

\PictureFrame{Epicycle}{Эпицикл движения Марса}
{Траектория движения Марса по эпициклу\\http://astro.unl.edu/classaction/animations/renaissance/marsorbit.html}

\begin{Person}{Kopernik}{Николай Коперник}{1473--1543}
\Citation{В центре находится Солнце. 

Земля и другие планеты вращаются вокруг Солнца и своей оси.

Луна вращается вокруг Земли.}{По <<О вращении небесных сфер>>, 1543}
\end{Person}

\begin{Person}{Luter}{Мартин Лютер}{1483--1562}
Люди слушают выскочку-астролога, который тщится показать, что вращается Земля, а не небеса
или небесный свод, Солнце и Луна. Всякий, кто желает казаться умнее, должен выдумать
какую-то новую систему, которая, конечно, из всех систем является самой лучшей. Этот
дурак хочет перевернуть всю астрономию, но Священное Писание говорит нам, что Иисус
Навин приказал остановиться Солнцу, а не Земле.
\end{Person}

\begin{Person}{Kepler}{Иоганн Кеплер}{1571--1630}
\Citation{\begin{enumerate}
\item Каждая планета Солнечной системы обращается по эллипсу, в одном из фокусов которого находится Солнце
\item Каждая планета движется в плоскости, проходящей через центр Солнца, причём за равные промежутки времени радиус-вектор, соединяющий Солнце и планету, описывает равные площади.
\item Квадраты периодов обращения планет вокруг Солнца относятся как кубы больших полуосей орбит планет.
\end{enumerate}}{По <<Новая астрономия>>, 1609}
\end{Person}



\begin{Person}{Bellarmine}{Кардинал Беллармино}{1542--1621}
Если сказать, что предположение о движении Земли и неподвижности Солнца позволяет 
представить все явления лучше, чем принятие эксцентриков и эпициклов, 
то это будет сказано прекрасно и не влечёт за собой никакой опасности. Для математика 
этого вполне достаточно. 

Но желать утверждать, что Солнце в действительности является центром мира и
 вращается только вокруг себя, не передвигаясь с востока на запад, 
что Земля стоит на третьем небе и с огромной быстротой вращается вокруг Солнца --
 утверждать это очень опасно; это значило бы нанести вред святой вере,
 представляя положения Святого Писания ложными.
\end{Person}

\end{document}