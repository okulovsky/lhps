\documentclass[aspectratio=169]{beamer}
\usepackage{framed}
\usepackage{lhps}

\renewcommand{\ImageSource}{L2-Images/}

\begin{document}

\EmptyPictureFrame{IndiaGood}

\EmptyPictureFrame{IndiaBad}

\EmptyPictureFrame{Krakow}

\EmptyPictureFrame{KrakowRain}


\begin{bframe}
\begin{Reason}
\from[1-]{Мы должны основать колонию на Марсе!}
\from[2-3]{Это слишком дорого! Если строить колонию, то не хватит денег на социальные программы}
\from[4-]{У того, кто строит колонию, не хватит денег на социальные программы}
\from[5-]{Колонию собирается строить тот же, чьей обязанностью является проводить социальные программы}
\from[6-]{Социальные программы должны быть}
\conc[3-]{Противоречие}
\end{Reason}
\end{bframe}

\begin{bframe}
\begin{Reason}
\from[1-]{Мы должны основать колонию на Марсе!}
\from[2-]{На Марсе может существовать экосистема}
\from[3-]{Колонизация может ее разрушить}
\from[4-]{Мы не должны разрушать экосистемы}
\conc[5-]{Противоречие}
\end{Reason}
\end{bframe}

\EmptyPictureFrame{avatar}

\begin{bframe}
\begin{Reason}
\from[3-]{Земля становится непригодной для жизни}
\from[4-]{Люди должны жить}
\conc[5-]{Нужно колонизировать другие планеты}
\from[6-]{Марс колонизировать проще всего}
\from[1]{Мы должны основать колонию на Марсе!}
\conc[2,7]{Мы должны основать колонию на Марсе!}
\end{Reason}
\end{bframe}

\begin{bframe}
\begin{Reason}
\from[1-]{Пусть колония на Марсе основана}
\from[2-]{Жить в ней смогут лишь избранные}
\from[3-]{Этих избранных не будет больше волновать пригодность Земли для жизни}
\from[4-]{Для максимизации прибыли эксплуатация земных ресурсов увеличится}
\conc[5-]{Земля будет становиться непригодной для жизни быстрее}
\conc[6-]{Оставшиеся на Земле не выживут}
\end{Reason}
\end{bframe}

\EmptyPictureFrame{elysium}
	
\begin{bframe}
\begin{Reason}
\from[8-]{GPS-навигация, тефлон, термобелье, <<липучки>> и <<молнии>> сделали жизнь лучше для всех}
\from[7]{GPS-навигация сделала жизнь лучше для всех}
\conc[6-]{Космические технологии в целом делают жизнь лучше для всех}
\from[5]{Космические технологии в целом делают жизнь лучше для всех}
\from[4]{Апелляция к Гильберту}
\conc[3-]{Технологии, разработанные для этой цели, сделают жизнь лучше для всех}
\from[2]{Технологии, разработанные для этой цели, сделают жизнь лучше для всех}
\conc[1-]{Мы должны основать колонию на Марсе!}
\end{Reason}
\end{bframe}

\begin{Person}{hilbert}{Давид Гильберт}{1862--1943}

Наиболее значимая для человечества задача... поймать муху на Луне. 

Для того чтобы поймать муху на Луне, придётся долететь до Луны, создать там искусственную атмосферу.. и только тогда названная цель может быть достигнута. 

Сколько удивительных открытий и изобретений придётся сделать, чтобы добиться названного результата!
\end{Person}


\begin{Person}{russell}{Бертран Рассел}{1872-1970}
\Citation{
\only<+>{
Многие верующие ведут себя так, словно не догматикам надлежит доказывать общепринятые постулаты, а наоборот -- скептики обязаны их опровергать. 

Это, безусловно, не так. Если бы я стал утверждать, что между Землей и Марсом вокруг Солнца по эллиптической орбите вращается фарфоровый чайник, никто не смог бы опровергнуть моё утверждение, добавь я предусмотрительно, что чайник слишком мал, чтобы обнаружить его даже при помощи самых мощных телескопов. 
}

\only<+>{
Но заяви я далее, что, поскольку моё утверждение невозможно опровергнуть, разумный человек не имеет права сомневаться в его истинности, то мне справедливо указали бы, что я несу чушь. 

Однако если бы существование такого чайника утверждалось в древних книгах, о его подлинности твердили каждое воскресенье... то неверие в его существование казалось бы странным, а сомневающийся -- достойным внимания психиатров в просвещённую эпоху, а ранее -- внимания инквизиции.
}
}{Из <<Есть ли Бог?>>, 1952}

\end{Person}

\begin{bframe}\frametitle{Исход обоснования тезиса}

\begin{itemize}
\item<+-> Общие принципы
\item<+-> Неспособность или нежелание построить очередное обоснование
\item<+-> Доказательство многословием
\item<+-> Принципы
\end{itemize} 

\end{bframe}

\begin{Person}{rand}{Айн Рэнд}{1905-1982}
\Citation{
--- Сеньор Д'Анкония, я не согласна с вами!

--- Если вы можете опровергнуть любой из моих доводов, мадам, я с благодарностью выслушаю вас.

--- О, я не могу. У меня нет конкретных возражений, мой мозг работает иначе, чем ваш, но я чувствую, что вы не правы, и поэтому знаю, что вы ошибаетесь.

--- Откуда вы это знаете?

--- Я это чувствую. Я живу не головой, а сердцем. Возможно, вы сильны в логике, но вы бессердечный человек.
}{Из <<Атлант расправил плечи>>, 1957}
\end{Person}

\begin{frame}\frametitle{Соломенный человек}
\begin{Reason}
\say{Джон}{Мы должны облегчить ограничения на продажу пива}
\say{Джейн}{Общество, которое позволяет свободный оборот одурманивающих веществ, потеряет нравственность, выродится и в конце концов погибнет!}
\end{Reason}
\end{frame}

\begin{Person}{flew}{Энтони Флю}{1923-2010}
\Citation{
Представьте себе Хамиша Макдональда, читающего статью про то как <<Брайтонский маньяк опять нападает>>. Хамиш шокирован и провозглашает: <<Ни один шотландец не сделал бы ничего подобного>>.

 На следующий день он натыкается на статью про какого-то мужика из Абердина, по сравнению с жестокостью которого брайтонский маньяк кажется просто джентльменом. Этот факт показывает, что Хамиш ошибался, но собирается ли он признать это? Не похоже. В этот раз он говорит: <<Ни один {\it истинный} шотландец не сделал бы ничего подобного.>>
}{Из <<Размышление о размышлении>>}
\end{Person}



\end{document}

