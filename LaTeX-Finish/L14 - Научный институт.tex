\documentclass[aspectratio=169]{beamer}

\usepackage{lhps}
\renewcommand{\ImageSource}{L5-Images/}

\begin{document}


\begin{bframe}\frametitle{Мотивы исследователя}
\begin{itemize}
\item<+-> Желание найти новое научное знание\uncover<+>{?}
\item<+-> Вера в иллюзии
\item<+-> Деньги
\item<+-> Слава
\end{itemize}
\end{bframe}

\EmptyPictureFrame{Vybegallo}


\begin{Person}{Lysenko}{Трофим Денисович Лысенко}{1898--1976}
\begin{itemize}
\item  гены - это выдумка генетиков, ведь их никто не видел
\item  практики не могут ждать тысячу лет, пока произойдет нужная им мутация, 
надо воспитывать растения и животных, в результате воспитания их наследственность 
быстро изменится в нужную сторону
\item разговоры про гены человека -- это основа расизма и фашизма
\item  работа на дрозофиле - это трата народных денег и вредительство, 
работать надо на коровах и овцах
\end{itemize}
\end{Person}


\begin{bframe}
\Citation{
Буржуазная печать широко разрекламировала новую науку — кибернетику. Эта модная лжетеория, выдвинутая группкой американских <<учёных>>, 
претендует на решение всех стержневых научных проблем и на спасение человечества от всех социальных бедствий. 
Кибернетическое поветрие пошло по разнообразным отраслям знания: физиологии, психологии, социологии, психиатрии, лингвистике и др. 
По утверждению кибернетиков, поводом к созданию их лженауки послужило сходство между мозгом человека и современными сложными машинами.
}
{
Ярошевский М. <<Кибернетика — ``наука'' мракобесов>>, <<Литературная газета>>, 1952
}
\end{bframe}

\begin{bframe}\frametitle{Написание научной статьи}
\begin{enumerate}
\item<+-> Наблюдения, эксперименты, обработка данных, гипотезы
\item<+-> Подготовка статьи в научный журнал
\item<+-> Рецензирование
\item<+-> Опубликование, выступление на конференции
\item<+-> Новые работы с подтверждением, опровержением, развитием
\end{enumerate}
\end{bframe}

\begin{frame}\frametitle{Распространение научного знания}
\uncover<+->{Первичные источники}
\begin{itemize}
\item<+-> Отчеты
\item<+-> Научные статьи
\end{itemize}
\uncover<+->{Вторичные источники}
\begin{itemize}
\item<+-> Монографии и диссертации
\item<+-> Обзоры
\item<+-> Учебники и энциклопедии
\end{itemize}
\uncover<+->{Массовое знание}
\begin{itemize}
\item<+-> Научно-популярные переложения
\item<+-> Массовая культура
\end{itemize}
\end{frame}

\begin{bframe}\frametitle{Наукометрика}
\begin{itemize}
\item Количество цитат
\item Индекс Хирша
\item Импакт-фактор журнала
\end{itemize}
\end{bframe}


\begin{bframe}\frametitle{Признаки издательства тщеславия}
\begin{itemize}
\item<+-> Взымание денег за публикацию
\item<+-> Восточноевропейские и азиатские издательства
\item<+-> Размытая тематика
\item<+-> Нереалистичные сроки рецензирования
\item<+-> Ужасный дизайн сайтов
\end{itemize}
\end{bframe}


\end{document}