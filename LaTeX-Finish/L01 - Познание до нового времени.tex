\documentclass[aspectratio=169]{beamer}
\usepackage{lhps}

\begin{document}

\renewcommand{\ImageSource}{L1-Images/}

\section{Почему важна философия познания?}

\PictureFrame{WorldEconomicHistory}{История мировой экономики в одном графике}{Среднедушевой доход по времени с X в. до н.э. по настоящее время в относительных единицах}

\begin{Timeline}{1730}{1910}
\tlitem{1769}{0}{Автомобиль}{Automobile}
\tlitem{1783}{-0.1}{Пароход}{SteamShip}
\tlitem{1804}{-0.15}{Паровоз}{SteamTrain}
\tlfitem{1807}{0}{ДВС}{CombustionEngine}{Двигатель внутреннего сгорания (1807)}

\tlitem{1783}{0.15}{Монгольфьер}{Aerostat}
\tlitem{1852}{0.2}{Дирижабль}{Zippellin}
\tlitem{1901}{0.2}{Самолет братьев Райт}{Aircraft}

\tlitem{1826}{0}{Фотоаппарат}{CameraObscura}
\tlitem{1891}{-0.2}{Киноаппарат}{Cinema}

\tlitem{1832}{0}{Телеграф}{Telegraph}
\tlitem{1860}{0.15}{Телефон}{Telephone}
\tlitem{1895}{0.15}{Радио}{Radio}

\tlitem{1834}{0.15}{Электродвигатель}{ElectricalEngine}
\tlitem{1840}{0.15}{Лампа накаливания}{LightBulb}

\tlitem{1738}{0}{Ластик}{Eraser}
\tlitem{1823}{-0.4}{Макинтош}{Mackintosh}
\tlfitem{1826}{-0.25}{Вулканизация}{Rubber}{Вулканизированная резина (1826)}

\tlitem{1797}{-0.15}{Фанера}{Pilewood}
\tlitem{1853}{-0.1}{Джинсы}{Jeans}
\tlitem{1894}{-0.07}{Кукурузные хлопья}{CornFlakes}
\end{Timeline}


\begin{GalleryFrame}
\image{BookPrinting}{Книгопечатный станок}
\image{Clocks}{Механические часы}
\image{Glasses}{Очки}
\end{GalleryFrame}

\EmptyPictureFrame{Explorer}


\section{Познание в первобытном мире}

\begin{Person}{Nongqawuse}{Нонкгавусе}{ок. 1840 -- 1898}
Коса должны уничтожить их посевы и убить свой скот; взамен духи сметут британцев в море, и тогда зернохранилища и хлева коса станут полнее, чем прежде
\end{Person}

\begin{Person}{Dawkins}{Ричард Докинз}{1941 -- н.в.}
\Citation
{
Мем -- это информационный объект (идея, поведение), который передается от человека к человеку в рамках культурной среды
}
{
Эгоистичный ген, 1976
}
\end{Person}

\section{Античность}

\begin{Person}{Thales}{Фалеc}{ок. 624-- ок. 546 до н.э.}
Все возникает из воды и в неё превращается
\end{Person}

\begin{Person}{Anaximander}{Анаксимандр}{ок. 640 -- ок. 546 до н.э.}

Земля парит в центре мира, ни на что не опираясь. Землю окружают исполинские трубчатые кольца-торы, наполненные огнём. В самом близком кольце, где огня немного, имеются небольшие отверстия — звёзды. Во втором кольце с более сильным огнём находится одно большое отверстие — Луна. В третьем, дальнем кольце, имеется самое большое отверстие, размером с Землю; сквозь него сияет самый сильный огонь — Солнце.
\end{Person}

\begin{Person}{Unknown}{Эвбулид}{IV BC}
То, что не потерял, то имеешь. Рогов ты не терял. Следовательно, у тебя есть рога
\end{Person}

\renewcommand{\ReasonWidth}{5cm}
\begin{Person}{Aristotle}{Аристотель}{384 -- 322 до н.э.}
\Citation{\begin{Reason}
\from{Все люди смертны}
\from{Сократ -- человек}
\conc{Сократ -- смертен}
\end{Reason}}{Органон}
\end{Person}

\PictureFrame{Aerolipile}{Античная паровая машина}{Паровая машина Герона Александрийского, I век н.э.}

\section{Темные века}

% http://www.youtube.com/watch?v=GylVIyK6voU

\begin{frame}[plain]
\begin{center}
\begin{tikzpicture}[y=-1cm]
\uncover<+->{\node at(0,-2)	{\includegraphics[scale=0.4, angle=5]{\ImageSource Zimbabwe1.png}};}
\uncover<+->{\node at(1,2)	{\includegraphics[scale=0.4, angle=-10]{\ImageSource  Zimbabwe2.png}};}
\uncover<+->{\node at(0,0)  {\includegraphics[scale=0.4]{\ImageSource Zimbabwe3.png}};}
\end{tikzpicture}
\end{center}
\end{frame}

\begin{Person}{Anaxagor}{Анаксагор}{ок. 428 -- ок. 510 до н.э.}

\only<1>{Солнце -- гигантский шар раскаленного металла}
\only<2>{Не я потерял Афины; Афины потеряли меня!}	

\end{Person}

\begin{Person}{Tertullianus}{Тертуллиан}{ок. 155 -- ок. 240}
\Citation{Верую, ибо абсурдно!}{De Carne Christi}
\end{Person}

\begin{Person}{Tertullianus}{Тертуллиан}{ок. 155 -- ок. 240}
Для того, чтобы осмеять Христа и заставить людей считать, что христиане лишь копируют веру в языческих богов, демоны стали вдохновителями мифологии. Демонам было заранее известно, чему будут учить христиане, и поэтому они измыслили сходные мифы и обряды и коварно разыграли их до евангельских событий
\end{Person}

\begin{Person}{Theodosius}{Феодосий II}{401 -- 450}
\Citation{
Запрещается изучать математику. Если кто-либо днем или ночью будет задержан в момент занятий (в частном порядке или в школе) этой запрещенной ложной дисциплиной, то оба должны быть преданы смертной казни.
}{
Кодекс Феодосия, 9.16.4., 438 г.
}
\end{Person}	

\begin{Person}{Gregory}{Григорий I}{540 -- 604}
Невежество -- мать истинного благочестия
\end{Person}


\end{document}