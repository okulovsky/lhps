\documentclass[aspectratio=169]{beamer}

\usepackage{lhps}
\renewcommand{\ImageSource}{L6-Images/}

\usepackage{pifont}




\begin{document}

\begin{Person}{Russo}{Жан-Жак Руссо}{1712--1778}

\uncover<+->{}

\Citation{
\begin{itemize}
\item<+-> Раньше было лучше
\item<+-> Человек сам по себе хорош. Плохим его делает общество
\item<+-> Неравенство -- это плохо
\item<+-> Корни наук и исскусств -- в пороках: тщеславии, роскоши
\end{itemize}
}
{
<<Рассуждение о науках и искусствах>> (1750)


<<Рассуждение о происхождении неравенства>> (1754)

}

\end{Person}

\EmptyPictureFrame{Existentialism}

\begin{bframe}\frametitle{Экзистенциальные вопросы}
\begin{itemize}
\item<+-> Смерть
\item<+-> Свобода и ответственность
\item<+-> Одиночество
\item<+-> Смысл жизни
\end{itemize}
\end{bframe}

\EmptyPictureFrame{davinchi}

\begin{Person}{freud}{Зигмунд Фрейд}{1856 -- 1939}
\footnotesize{
\begin{tabular}{c c c c}
& \uncover<3->{Ребенок} & \uncover<3->{Взрослый} & \uncover<3->{Родитель} \\
& \uncover<3->{Оно} & \uncover<3->{Я} & \uncover<3->{Сверх-Я} \\
& \uncover<2->{Ид} & \uncover<2->{Эго} & \uncover<2->{Супер-Эго} \\
\uncover<1->{Подсознательное} & \uncover<4->{\normalsize\checkmark} &\uncover<4->{\normalsize\ding{55}} & \uncover<4->{\normalsize\checkmark}\\
\uncover<1->{Сознательное} & \uncover<4->{\normalsize\ding{55}} & \uncover<4->{\normalsize\checkmark} &  \uncover<4->{\normalsize\checkmark}
\end{tabular}}
\end{Person}

\begin{bframe}\frametitle{Защитные механизмы психики}
\begin{itemize}
\item<+-> Вытеснение
\item<+-> Проекция
\item<+-> Рационализация
\item<+-> ...
\item<+-> Сублимация
\end{itemize}

\end{bframe}

\section{Первая мировая война}

\begin{bframe}
\begin{itemize}
\item<+-> Вера в светлое будущее\uncover<+->{?}
\item<+-> Вера в разум?
\item<+-> Вера в науку?
\end{itemize}
\end{bframe}

\section{Вторая мировая война}


\begin{frame}

$ $\\[1cm]

\begin{center}
\begin{tikzpicture}[x=3.5cm,y=-3cm]

\node at (0,0) { \begin{tabular}{l} {\bf Общество модерна} \\ Рационализм \\ Либерализм \\ Капитализм \end{tabular} };

\uncover<2->{\node at (-1,1) { \begin{tabular}{l} {\bf Социализм/Коммунизм} \\ Эгалитаризм, коллективизм \\ Гиперрациональность \\ Ограничение частной собственности  \end{tabular} };}
 
\uncover<3->{\node at (1,1)  { \begin{tabular}{l} {\bf Фашизм}  \\  Элитаризм, иерархичность \\ Иррационализм, культ традиции \\ Вождизм, милитаризм, поиски врагов  \\ \end{tabular} };}


\end{tikzpicture}
\end{center}
\end{frame}



\begin{Person}{Popper}{Карл Поппер}{1902--1994}
\Citation{
Открытое общество -- общество, где люди вынуждены принимать личные решения.

Закрытое общество -- племенное, магическое или коллективистское.
}
{<<Открытое общество и его враги>>, 1945}
\end{Person}



\begin{ImageAndText}{AdornoHorkheimer}{Макс Хоркхаймер (1895--1973) и Теодор Адорно (1903--1969)}

\Citation{
\begin{itemize}
\item<+-> Причина катастрофы XX века -- в модерне и премодерне
\item<+-> Тройное доминирование
\begin{itemize}
\item<+-> Доминирование человека над окружающей природой
\item<+-> Доминирование человека над собственной природой 
\item<+-> Доминирование человека над человеком
\end{itemize}
\end{itemize}
}
{
<<Диалектика Просвещения>>, 1947
}
\end{ImageAndText}


\begin{bframe}\frametitle{После второй мировой войны}
\begin{itemize}
\item<+-> Корейская война
\item<+-> Берлинский кризис
\item<+-> Карибский кризис
\end{itemize}
\end{bframe}

\EmptyPictureFrame{DuckAndCover}





\begin{Person}{Fukuyama}{Френсис Фукуяма}{1952 -- н.в.}

\Citation{
Триумф Запада, западной идеи очевиден прежде всего потому, что у либерализма не осталось никаких жизнеспособных альтернатив... То, чему мы, вероятно, свидетели, -- не просто конец холодной войны или очередного периода послевоенной истории, но конец истории как таковой, завершение идеологической эволюция человечества и универсализации западной либеральной демократии как окончательной формы правления.
}
{<<Конец истории>>, 1989}

\end{Person}

\end{document}