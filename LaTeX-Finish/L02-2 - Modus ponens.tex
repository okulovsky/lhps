\documentclass[aspectratio=169]{beamer}
\usepackage{framed}
\usepackage{lhps}

\newcommand{\task}{
В каждой комнате или принцесса, или тигр.

Если в первой комнате принцесса, то табличка на ней истинна, если тигр -- ложна. Во второй комнате наоборот.

\begin{columns}
	\column{2.5cm}
	\begin{framed}
		\begin{center}
			I
			
			Что выбрать -- большая разница!
		\end{center}
	\end{framed}
	\column{2.5cm}
	\begin{framed}
		\begin{center}
			II
			
			Лучше выбрать другую комнату.
		\end{center}
	\end{framed}
\end{columns}	
}


\begin{document}

\renewcommand{\ImageSource}{L2-Images/}
\newcommand{\firstcircle}{(-1,0) circle (1.5)}
\newcommand{\secondcircle}{(1,0) circle (1.5)}
\newcommand{\thirdcircle}{(0,1.73) circle (1.5)}

\section{Причинно-следственная логика}

\begin{frame}\frametitle{Правила вывода}
\begin{center}

\uncover<1->{\begin{Reason}
	\from{Если идет дождь, земля мокрая}
	\from{Сейчас идет дождь}
	\conc{Сейчас земля мокрая}
\end{Reason}}

\uncover<2->{
\begin{Reason}
	\from{Если идет дождь, земля мокрая}
	\from{Земля не мокрая}
	\conc{Дождь не идет}
\end{Reason}
}
\end{center}
\end{frame}

\begin{frame}
\begin{center}	
\begin{tikzpicture}
\uncover<1->{
\node[align=left] (n)  at (0,0) {
\from{Если идет дождь, земля мокрая}\\
\from{Дождь не идет}\\
\conc{Земля не мокрая}
};
}
\uncover<2->{
\draw[color=accent, ultra thick] (n.north west) -- (n.south east);
\draw[color=accent, ultra thick] (n.north east) -- (n.south west);
};
\end{tikzpicture}

$$
\ 
$$

\begin{tikzpicture}
\uncover<3->{
\node[align=left] (n)  at (0,0) {
\from{Если идет дождь, земля мокрая}\\
\from{Земля мокрая}\\
\conc{Идет дождь}
};
}
\uncover<4->{
\draw[color=accent, ultra thick] (n.north west) -- (n.south east);
\draw[color=accent, ultra thick] (n.north east) -- (n.south west);
};

\end{tikzpicture}

\end{center}
\end{frame}

\begin{frame}[plain]
\begin{columns}
\column{0.5\textwidth}
\task


\column{0.5\textwidth}
\uncover<2->{Если в первой комнате тигр, то утверждение I ложно.}\uncover<3->{ Тогда во второй тоже тигр. Тогда утверждение II истинно.}\uncover<4->{ Приходим к противоречию.}

\vskip1cm

\uncover<5->{Если в первой комнате принцесса, то утверждение I истинно. Тогда во второй комнате тигр. Тогда утверждение II истинно. Противоречий нет.}

\end{columns}
\end{frame}

\EmptyPictureFrame{SchrodingerCat}

\EmptyPictureFrame{Step}

\begin{bframe}
Отношение $\rho$ называется транзитивным в случае, когда для любых $x$, $y$, $z$ выполняется условие: если $x\rho y$ и $y\rho z$, то $x\rho z$.
\end{bframe}


\begin{frame}[plain]
\begin{columns}
\column{0.5\textwidth}

\task

\column{0.5\textwidth}

\begin{tabular}{l| l}
\uncover<2->{В первой комнате принцесса & $P_1$ \\ \hline 
}\uncover<3->{В первой комнате тигр & $\neg P_1$ \\ \hline 
}\uncover<4->{Во второй комнате принцесса & $P_2$ \\ \hline 
}\uncover<5->{Во второй комнате тигр & $\neg P_2$ \\ \hline 
}\uncover<6->{Если $A$, то $B$ & $A\rightarrow B$ \\ \hline 
}\uncover<7->{Верно и $A$, и $B$ & $A\wedge B$ \\ \hline
}\uncover<8->{Верно $A$, или $B$, или оба & $A\vee B$}\end{tabular}

$$
\uncover<10->{P_1\rightarrow}\uncover<11->{(P_1\wedge \neg P_2)\vee (\neg P_1 \wedge P_2)}
$$ $$
\uncover<12->{\neg P_1\rightarrow}\uncover<13->{(P_1\wedge P_2)\vee (\neg P_1 \wedge \neg P_2)}
$$ $$
\uncover<14->{P_2\rightarrow}\uncover<15->{\neg P_1 \vee P_2}
$$ $$
\uncover<16->{\neg P_2\rightarrow}\uncover<17->{P_1 \wedge \neg P_2}
$$ 

\end{columns}
\end{frame}


\begin{frame}[plain]
\begin{columns}
\column{0.5\textwidth}


\begin{tabular}{l| l}
В первой комнате принцесса & $P_1$ \\ \hline 
В первой комнате тигр & $\neg P_1$ \\ \hline 
Во второй комнате принцесса & $P_2$ \\ \hline 
Во второй комнате тигр & $\neg P_2$ \\ \hline 
Если $A$, то $B$ & $A\rightarrow B$ \\ \hline 
Верно и $A$, и $B$ & $A\wedge B$ \\ \hline
Верно $A$, или $B$, или оба & $A\vee B$ \\ 
\end{tabular}

$$
P_1\rightarrow(P_1\wedge \neg P_2)\vee (\neg P_1 \wedge P_2)
$$ $$
\neg P_1\rightarrow(P_1\wedge P_2)\vee (\neg P_1 \wedge \neg P_2)
$$ $$
\neg P_2\rightarrow P_1 \wedge \neg P_2
$$ $$
P_2\rightarrow \neg P_1 \vee P_2
$$

\column{0.5\textwidth}

$$
\uncover<2->{A,\ A\rightarrow B\ \Rightarrow\ B}
$$ $$
\uncover<3->{A\wedge B\ \Rightarrow\ A} \uncover<4->{,\ \ A\wedge B\ \Rightarrow\ B}
$$ \only<4-14>{$$
\uncover<4->{\neg P_1 \Rightarrow}\uncover<6->{(P_1\wedge P_2)\vee (\neg P_1 \wedge \neg P_2)}
$$ $$
\uncover<7->{\neg P_1, P_1\wedge P_2\Rightarrow}\uncover<8->{P_1\Rightarrow}\uncover<9->{\times}
$$ $$
\uncover<10->{\neg P_1, \neg P_1\wedge \neg P_2\Rightarrow}\uncover<11->{\neg P_2\Rightarrow}
$$ $$
\uncover<12->{\Rightarrow P_1 \wedge \neg P_2\Rightarrow }\uncover<13->{P_1 \Rightarrow}\uncover<14->{\times}
$$}\only<15->{$$
\uncover<15->{\neg P_1\Rightarrow \times}
$$ $$
\uncover<16->{P_1 \Rightarrow}\uncover<17->{(P_1\wedge \neg P_2)\vee (\neg P_1 \wedge P_2) \Rightarrow} $$ $$ 
\uncover<18->{\Rightarrow P_1\wedge \neg P_2 \Rightarrow}\uncover<19->{\neg P_2\Rightarrow}\uncover<20->{P_1\wedge\neg P_2}
$$}
\end{columns}
\end{frame}

\begin{frame}[plain]
\begin{columns}
\column{0.4\textwidth}
$$
P_1\rightarrow(P_1\wedge \neg P_2)\vee (\neg P_1 \wedge P_2)
$$ $$
\neg P_1\rightarrow(P_1\wedge P_2)\vee (\neg P_1 \wedge \neg P_2)
$$ $$
\neg P_2\rightarrow P_1 \wedge \neg P_2
$$ $$
P_2\rightarrow \neg P_1 \vee P_2
$$

$$
\begin{array}{c c c c c}
 \uncover<2-> {A & 0 & 0 & 1 & 1 \\
			  B & 0 & 1 & 0 & 1 \\ \hline\hline
}\uncover<3-> {\neg A & 1 & 1 & 0 & 0 \\
}\uncover<4-> {\neg B & 1 & 0 & 1 & 0\\
}\uncover<5-> {A \wedge B & 0 & 0 & 0 & 1 \\
}\uncover<6-> {A \vee  B & 0 & 1 & 1 & 1 \\
}\uncover<7-> {A \rightarrow B & 1 & 1 & 0 & 1 \\
}
\end{array}
$$

\column{0.6\textwidth}
$$
\begin{array}{c c c c c}
 \uncover<8->{P_1 & 0 & 0 & 1 & 1\\
P_2 & 0 & 1 & 0 & 1 \\ \hline\hline
P_1 \wedge P_2 & 0 & 0 & 0 & 1 \\
}\uncover<9->{P_1 \wedge \neg P_2 \uncover<10->{ & 0 & 0 & 1 & 0 }\\
}\uncover<11->{\neg P_1 \wedge P_2 \uncover<12->{& 0 & 1 & 0 & 0 }\\
}\uncover<13->{\neg P_1 \wedge \neg P_2\uncover<14->{ & 1 & 0 & 0 & 0 }\\
}\uncover<15->{\neg P_1 \vee P_2 \uncover<16->{& 1 & 1 & 0 & 1 }\\ \hline\hline
}\uncover<17->{(P_1 \wedge \neg P_2) \vee (\neg P_1 \wedge P_2) \uncover<18->{& 0 & 1 & 1 & 0 }\\
}\uncover<19->{(P_1 \wedge P_2) \vee (\neg P_1 \wedge \neg P_2) \uncover<20->{& 1 & 0 & 0 & 1} \\ \hline\hline
}\uncover<21->{P_1\rightarrow(P_1\wedge \neg P_2)\vee (\neg P_1 \wedge P_2) \uncover<22->{& 1 & 1 & 1 & 0} \\
}\uncover<23->{\neg P_1\rightarrow(P_1\wedge P_2)\vee (\neg P_1 \wedge \neg P_2)\uncover<24->{ & 1 & 0 & 1 & 1 }\\
}\uncover<25->{\neg P_2\rightarrow P_1 \wedge \neg P_2 \uncover<26->{& 0 & 1 & 1 & 1}\\
}\uncover<27->{P_2\rightarrow \neg P_1 \vee P_2 \uncover<28->{& 1 & 1 & 1 & 1 }\\
}\end{array}
$$
\end{columns}
\end{frame}


\end{document}