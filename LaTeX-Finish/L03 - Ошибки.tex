\documentclass[aspectratio=169]{beamer}
\usepackage{framed}
\usepackage{lhps}

\renewcommand{\ImageSource}{L2-Images/}
\begin{document}

\begin{Person}{Witgenstein}{Людвиг Витгенштейн}{1889--1951}
\Citation{
Значение слова есть его употребление в рамках языковой игры, а правила такой игры есть практика
}{
Философские исследования, 1953 г.
}
\end{Person}

\renewcommand{\ReasonWidth}{9cm}

\begin{frame}\frametitle{Пресуппозиции}
\begin{Reason}
	\from[1-]{Свет не горит}
	\from[2-]{Свет не горит, когда нет электричества или перегорела лампочка}
	\conc[2-]{Нет электричества или перегорела лампочка}
	\from[3-]{Электричество есть}
	\conc[3-]{Перегорела лампочка}
	\from[4-]{Если лампочка перегорела, то, если хочешь, чтобы появился свет, нужно поменять лампочку}
	\conc[5-]{Если хочешь, чтобы появился свет, нужно поменять лампочку}
	\from[6-]{Я хочу, чтобы появился свет}
	\conc[1,7]{Нужно поменять лампочку}
\end{Reason}
\end{frame}

\begin{frame}\frametitle{Модальности}
\begin{Reason}
\from[1-3]{Ежедневное употребление черники продлевает жизнь}
\from[4]{Ежедневные употребление дождевых червей продлевает жизнь}
\from[3-]{Я хочу продлить свою жизнь}
\conc[2-3]{Нужно ежедневно употреблять чернику}
\conc[4]{Нужно ежедневно употреблять дождевых червей}
\end{Reason}
\end{frame}

\begin{frame}\frametitle{Апелляции к авторитету}
\begin{Reason}
	\from[1-3]{Потребление сладкого неизбежно приводит к кариесу}
	\from[4-5]{Британские ученые считают, что потребление сладкого неизбежно приводит к кариесу}
	\conc[5]{Потребление сладкого неизбежно приводит к кариесу}
	\from[2-5]{Кариес для меня неприемлем}
	\conc[3-5]{Я должен воздержаться от потребления сладкого}
	\from[6]{Британские ученые считают, что X}
	\conc[6]{X}
		
\end{Reason}
\end{frame}

\begin{frame}\frametitle{Апелляции к авторитету}
	\begin{Reason}
		\from{X утверждает Y}
		\from{X обладает качеством Z}
		\conc{Y -- истинно}
	\end{Reason}
\end{frame}



\begin{frame}\frametitle{Ad hominem}

\Citation{<<Какие взгляды на архитектуру может высказать мужчина без прописки? Пойманный с поличным, он сознается и признает себя побеждённым. И вообще, разве нас может интересовать мнение человека лысого, с таким носом? Пусть сначала исправит нос, отрастит волосы, а потом и выскажется>>}{Михаил Жванецкий}
\end{frame}

\begin{frame}
\begin{Reason}
\from{X есть Y}
\from{X обладает качеством Z}
\conc{Y -- ложно}
\end{Reason}
\end{frame}

\begin{frame}\frametitle{Ad hominem}
	\begin{Reason}
		\from[1-]{X утверждает, что соседи облучают его через розетку}
		\from[2-]{X -- пациент психиатрической больницы с бредовыми идеями}
		\conc[3-]{Соседи не облучают X через розетку}
	\end{Reason}
\end{frame}



\EmptyPictureFrame{cancer}

\begin{frame}\frametitle{Оценочные суждения}
\begin{Reason}
	\from[1-]{Есть раков -- отвратительно}
	\from[2]{Не следует делать отвратительных вещей}
	\conc[1-]{Не следует есть раков}
\end{Reason}
\end{frame}

\begin{frame}\frametitle{Защитные термины}
\begin{Reason}
	\from[1]{Все тела падают на землю}
	\from[2]{Почти все тела падают на землю}
	\from[1-]{Камень -- это тело}
	\conc[1-]{Камень упадет на землю}
\end{Reason}
\end{frame}

\begin{frame}\frametitle{Ambiguity}
\end{frame}

\begin{frame}\frametitle{Циклические рассуждения}
\Citation{
\say[+-]{Алиса}{Откуда я знаю, что вам можно доверять?}

\say[+-]{Джон}{Вот поручительство Боба.}

\say[+-]{Алиса}{Но почему мне следует доверять Бобу?}

\say[+-]{Джон}{Уверяю вас, он честнейший человек!}
	
}{}
\end{frame}

\begin{frame}\frametitle{Циклические рассуждения}
\begin{Reason}
	\from[1-]{Если X доверяет Y, а Y доверяет Z, то X должен доверять Z}
	\from[2-]{Боб доверяет Джону}
	\conc[3-]{Если Алиса доверяете Бобу, то Алиса должна доверять Джону}
	\from[4-]{Джон доверяет Бобу}
	\from[5-]{Если A доверяет B, а B доверяет C, то A должен доверять C}
	\conc[6-]{Если Алиса доверяет Джону, то Алиса должна доверять Бобу}
	\conc[7-]{Если Алиса доверяет Джону, то Алиса доверяет Джону}
\end{Reason}
\end{frame}

\begin{frame}\frametitle{Парадокс кучи}
\begin{Reason}
	\from[1-]{У Джона нет бороды}
	\from[2-]{Если у человека нет бороды, то за один день без бритья бороды не появится}
	\conc[3-]{У Джона завтра не будет бороды}
	\from[4-]{Если у человека нет бороды, то за один день без бритья бороды не появится}
	\conc[5-]{У Джона послезавтра не будет бороды}
	\from[6-]{...}
	\conc[7-]{У Джона никогда не будет бороды}
\end{Reason}
\end{frame}

\begin{frame}\frametitle{Парадокс кучи}
\begin{Reason}
	\from[1]{Не существует самого большого четного числа}	\from[2-]{Число $2$ четно}
	\from[3-]{Если $X$ четно, то $X$+2 больше $X$ и четно}
	\conc[4-]{Число $4$ четно и больше $2$}
	\from[5-]{Если $X$ четно, то $X$+2 больше $X$ и четно}
	\conc[6-]{Число $6$ четно и больше $4$}
	\from[7-]{...}
	\conc[8-]{Для любого четного числа $X$ есть четное число, большее его и четное}
	\conc[8-]{Не существует самого большого четного числа}
\end{Reason}
\end{frame}

\end{document}