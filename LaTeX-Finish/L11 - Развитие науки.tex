\documentclass[aspectratio=169]{beamer}
\usepackage{framed}
\usepackage{lhps}

\renewcommand{\ImageSource}{L3-Images/}
\begin{document}

\begin{Person}{Ptolemy}{Клавдий Птолемей}{ок. 90 -- ок. 168}
\Citation{
\begin{itemize}
\item Земля находится в центре
\item Солнце и Луна вращаются вокруг Земли по круговым орбитам
\item Планеты вращаются вокруг Земли по эпициклам
\end{itemize}
}{<<Альмагест>>, 140}
\end{Person}

\PictureFrame{Epicycle}{Эпицикл движения Марса}
{Траектория движения Марса по эпициклу\\http://astro.unl.edu/classaction/animations/renaissance/marsorbit.html}

\begin{Person}{Kopernik}{Николай Коперник}{1473--1543}
\Citation{
\begin{itemize}
\item В центре находится Солнце
\item Земля и другие планеты вращаются вокруг Солнца и своей оси
\item Луна вращается вокруг Земли
\end{itemize}
}{<<О вращении небесных сфер>>, 1543}
\end{Person}

\begin{Person}{Luter}{Мартин Лютер}{1483--1562}
Люди слушают выскочку-астролога, который тщится показать, что вращается Земля, а не небеса
или небесный свод, Солнце и Луна. Всякий, кто желает казаться умнее, должен выдумать
какую-то новую систему, которая, конечно, из всех систем является самой лучшей. Этот
дурак хочет перевернуть всю астрономию, но Священное Писание говорит нам, что Иисус
Навин приказал остановиться Солнцу, а не Земле.
\end{Person}


\begin{Person}{Bellarmine}{Кардинал Беллармино}{1542--1621}
Если сказать, что предположение о движении Земли и неподвижности Солнца позволяет 
представить все явления лучше, чем принятие эксцентриков и эпициклов, 
то это будет сказано прекрасно и не влечёт за собой никакой опасности. Для математика 
этого вполне достаточно. 

Но желать утверждать, что Солнце в действительности является центром мира и
 вращается только вокруг себя, не передвигаясь с востока на запад, 
что Земля стоит на третьем небе и с огромной быстротой вращается вокруг Солнца --
 утверждать это очень опасно; это значило бы нанести вред святой вере,
 представляя положения Святого Писания ложными.
\end{Person}

\begin{Person}{Kepler}{Иоганн Кеплер}{1571--1630}
\Citation{\begin{enumerate}
\item Каждая планета Солнечной системы обращается по эллипсу, в одном из фокусов которого находится Солнце
\item Каждая планета движется в плоскости, проходящей через центр Солнца, причём за равные промежутки времени радиус-вектор, соединяющий Солнце и планету, описывает равные площади.
\item Квадраты периодов обращения планет вокруг Солнца относятся как кубы больших полуосей орбит планет.
\end{enumerate}}{<<Новая астрономия>>, 1609}
\end{Person}

\begin{Person}{Galilei}{Галилео Галилей}{1564 -- 1642}
\begin{itemize}
\item Горы на Луне
\item Пятна на Солнце
\item Спутники Юпитера
\end{itemize}
\end{Person}


\begin{Person}{Aristotle}{Аристотель}{384 -- 322 до н.э.}
\Citation{
\begin{itemize}
\item Скорость падения пропорциональна весу тела.
\item Для движения к телу должна быть приложена внешняя сила
\end{itemize}
}{<<Физика>>}
\end{Person}

\begin{Person}{Galilei}{Галилео Галилей}{1564 -- 1642}
\Citation{
\begin{itemize}
\item Скорость падения не зависит от веса тела
\item В отсутствии внешних сил тело покоиться или двигаться с постоянной скоростью
\item Если в двух замкнутых лабораториях, одна из которых равномерно прямолинейно (и поступательно) движется относительно другой, провести одинаковый механический эксперимент, результат будет одинаковым.
\end{itemize}
}{<<Беседы и математические доказательства, касающиеся двух новых отраслей науки>>, 1638}
\end{Person}

\begin{Person}{Newton}{Исаак Ньютон}{1643 -- 1727}
\Citation{
\only<1>{\begin{enumerate}
\item Всякое тело продолжает удерживаться в состоянии покоя или равномерного и прямолинейного движения, пока и поскольку оно не понуждается приложенными силами изменить это состояние.

\item Изменение количества движения пропорционально приложенной движущей силе

\item Действию всегда есть равное и противоположное противодействие
\end{enumerate}}
\only<2>{
Любые тела участвуют в гравитационном взаимодействии, притягивая друг друга с силой, пропорциональной произведению масс и обратно пропорциональной квадрату расстояния между ними.
}
}{<<Математические начала натуральной философии>> (1687)}
\end{Person}

\newcommand{\script}[3]
{
#1 (#2) \\ #3
}

\begin{Person}{Descartes}{Рене Декарт}{1596--1650}
Сомнения в:

\begin{itemize}
\item философах и авторитетах
\item опыте и ощущениях
\item логике
\end{itemize}
\end{Person}

\begin{Person}{Descartes}{Рене Декарт}{1596--1650}
\Citation{
\only<+>{
Так как мы рождаемся детьми и составляем разные суждения о вещах прежде, чем достигнем полного употребления своего разума, то многие предрассудки отклоняют нас от познания истины; избавиться от них мы, по-видимому, можем не иначе, как постаравшись раз в жизни усомниться во всём том, в чём найдём хотя бы малейшее подозрение недостоверности.}

\only<+->{ Если мы станем отвергать всё то, в чём каким бы то ни было образом можем сомневаться, и даже будем считать всё это ложным, то хотя мы легко предположим, что нет никакого Бога, никакого неба, никаких тел и что у нас самих нет ни рук, ни ног, ни вообще тела, однако же не предположим также и того, что мы сами, думающие об этом, не существуем: ибо нелепо признавать то, что мыслит, в то самое время, когда оно мыслит, не существующим. Вследствие чего это познание: я мыслю, следовательно существую, — есть первое и вернейшее из всех познаний}
}
{<<Первоначала философии>>, 1644}
\end{Person}

\begin{bframe}
\begin{center}
Всякое тело продолжает удерживаться в состоянии покоя или равномерного и прямолинейного движения, пока и поскольку оно не понуждается приложенными силами изменить это состояние.
\end{center}
\end{bframe}

\begin{Person}{Bacon}{Фрэнсис Бэкон}{1561--1626}
\Citation{
Эмпирики,  подобно  муравью, только  собирают  и  довольствуются  собранным.
Рационалисты,  подобно  паукам,  производят ткань  из самих  себя. Пчела  же
избирает средний способ: она извлекает материал из садовых и полевых цветов,
но располагает и изменяет  его  по своему умению.  Не отличается от этого  и
подлинное дело философии.
}{
Новый Органон, 1620 г.
}
\end{Person}

\begin{Person}{Locke}{Джон Локк}{1632--1704}
\begin{itemize}
\item Опыт является единственным источником знания
\item Не нужно рассматривать понятия, которые не прослеживаются до опыта
\end{itemize}
\end{Person}


\begin{Person}{Hume}{Дэвид Юм}{1711--1776}
\Citation{
Мы не способны, исходя из одного примера, открыть какую-либо силу связывающую действие с причиной и делающее первое неизменным следствием второй. Мы находим только, что действие в самом деле, фактически, следует за причиной. Толчок, производимый одним бильярдным шаром, сопровождается движением второго. И это все, что является внешним чувствам. Ни в каком единичном, частном случае причинности нет ничего такого, что могло бы вызвать идею силы, или необходимой связи.
}{<<Исследование о человеческом разумении>>, 1748}

\end{Person}

\begin{bframe}
\begin{center}
\begin{tabular}{c c c c}
&& \uncover<3->{Аналитические} 	
& \uncover<3->{Синтетические} \\

& & \\[-0.2cm] 

  \uncover<2->{Априорные} 		
&& \uncover<4->{Силлогизм}
& \uncover<7->{Закон Ньютона}\uncover<8>{?} \\

  \uncover<2->{Апостериорные}
&& \uncover<6->{---}
& \uncover<5->{Эмпирический факт} \\
\end{tabular}
\end{center}
\end{bframe}



\begin{Person}{Kant}{Иммануил Кант}{1724--1804}
\begin{center}
\begin{tikzpicture}[x=2cm,y=-3cm]

\node[draw=black,align=center] (n0) at (0,0) {Реальный мир\\(вещь в себе)};
\node[draw=black] (n1) at (0,1) {Феномен};
\draw[->] (n0) -- node[right, align=center] {восприятие \\ в категориях} (n1);

\end{tikzpicture}
\end{center}

\begin{flushright}
<<Критика чистого разума>>, 1781
\end{flushright}


\end{Person}


\end{document}