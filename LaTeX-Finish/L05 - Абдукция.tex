\documentclass[aspectratio=169]{beamer}
\usepackage{framed}
\usepackage{lhps}

\renewcommand{\ImageSource}{L2-Images/}
\begin{document}

\section{Причинно-следственное мышление}

\begin{bframe}
\begin{Reason}
	\from{Если идет дождь, то земля становится мокрой}
	\from{Идет дождь}
	\conc{Земля мокрая}
\end{Reason}
\end{bframe}

\EmptyPictureFrame{Johnny}

\begin{bframe}
\begin{Reason}
	\from[3]{Если идет дождь, то капли воды находятся в воздухе}
	\from[3]{Если капли воды находятся в воздухе, то они падают на землю}
	\from[3]{Если много капель воды падает на землю, то земля смешивается с водой}
	\from[3]{Если земля смешивается с водой, то земля становится мокрой}
	\conc[2-]{Если идет дождь, то земля становится мокрой}
	\from[1]{Если идет дождь, то земля становится мокрой}
	\from[1-]{Идет дождь}
	\conc[1-]{Земля мокрая}
\end{Reason}
\end{bframe}



\begin{bframe}
\begin{Reason}
	\from[7-]{Если тело тяжелее воздуха, то оно падает на землю}
	\conc[6-]{\bf Ну, существуют силы гравитации...}
	\from[5]{\bf Ну, существуют силы гравитации...}
	\conc[4-]{Если тело тяжелее воздуха, то оно падает на землю}
	\from[3]{Если тело тяжелее воздуха, то оно падает на землю}
	\from[3-]{Капли воды тяжелее воздуха}
	\conc[2-]{Если капли воды находятся в воздухе, то они падают на землю}
	\from[1]{Если капли воды находятся в воздухе, то они падают на землю}
\end{Reason}
\end{bframe}

\begin{bframe}
\begin{Reason}
\from[3-]{Валерьана содержит актинидин}
\from[3-]{Актинидин имеет запах, схожий с  3-меркапто-3-метилбутан-1-олом}
\from[3-]{3-меркапто-3-метилбутан-1-ол содержится в моче кошачьих}
\from[4]{\bf Ну, наверное, это вещество как-то особенно действует на нервную систему...}
\conc[2-]{Если коту дать валерьану, он начнет вести себя необычно}
\from[1]{Если коту дать валерьану, он начнет вести себя необычно}
\end{Reason}
\end{bframe}

\section{Классификация причинно-следственных связей}

\begin{frame}

\uncover<+->{
$A$ {\bf необходимо} для $B$, если $B$ всегда предшествует $A$. При наличии $A$ может не наблюдаться $B$.\\[0.2cm]
}

\uncover<+->{
Включение выключателя необходимо для того, чтобы в настольной лампе появился свет.\\[0.5cm]
}

\uncover<+->{
$A$ {\bf достаточно} для $B$, если $A$ всегда сопровождается $B$. При наличии $B$ может не наблюдаться $A$.\\[0.2cm]
}

\uncover<+->{
Сильный ветер достаточен для того, чтобы качались деревья\\[0.5cm]
}

\uncover<+->{
$A$ {\bf способствует} $B$, если $A$ предшествует $B$ и изменение $A$ сопровождается изменением $B$. При наличии $A$ может не быть ни необходимым, ни достаточным. \\[0.2cm]
}

\uncover<+->{
Экономия способствует накоплению богатств.
}
\end{frame}

\begin{frame}
\begin{center}
\begin{tikzpicture}[x=4cm,y=1cm]
\node[align=center] (n1) at (0,-1) {\only<1>{Переохлаждение}\only<2>{Сниженный\\иммунитет}};
\node (n2) at (0,1) {Инфекция};
\node (n3) at (1,0) {Простуда};
\draw[->] (n2) -- (n3);
\draw[->] (n1) -- (n3);

\uncover<2>{
\node (n4) at (-1,0) {Переохлаждение};
\node (n5) at (-1,-2) {Отсутствие прививки};
\draw[->] (n5) -- (n1);
\draw[->] (n4) -- (n1);
}
\end{tikzpicture}
\end{center}
\end{frame}

\begin{frame}
\begin{center}
\begin{tikzpicture}[x=4cm,y=1cm]

\node (n1) at (0,0) {Бег};
\node (n2) at (1,1) {Сердцебиение};
\node (n3) at (1,-1) {Усталось};

\draw[->] (n1) -- (n2);
\draw[->] (n1) -- (n3);


\end{tikzpicture}
\end{center}

\end{frame}

\begin{frame}
\begin{center}
\begin{tikzpicture}[x=3cm,y=2cm]

\uncover<+->{
\node[align=center] (n1) at (-1,0) {Низкоквалифицированная\\работа};
}

\uncover<+->{

\node[align=center] (n2) at (1,0) {Бедность};
\draw[->] (n1) -- (n2);
}
\uncover<+->{

\node[align=center] (n3) at (0,-1) {Плохое \\ образование};
\draw[->] (n2) -- (n3);
}
\uncover<+->{
\draw[->] (n3) -- (n1);
}




\end{tikzpicture}
\end{center}
\end{frame}

\section{Ошибки причинно-следственного мышления}


\PictureFrame{Correlation}{Корреляция}{\footnotesize Корреляция между возрастом мисс Америки и количеством убийств паром}

\PictureFrame{Correlation1}{Корреляция}{\footnotesize Корреляция между импортом нефти в США и количеством погибших в столкновении с поездами}

%Глава заканчивается почти анекдотическим (но реальным) примером перепутывания причины и следствия аборигенами Новых Гебрид. Они полагали, что наличие вшей ведёт к здоровью. Этот вывод делался на том основании, что больного человека вши покидали (так как вследствие повышенной температуры тела условия существования для них становились некомфортными), тогда как у всех здоровых людей они были (иными словами, наблюдалась положительная корреляция между здоровьем и наличием вшей).

\section{Абдукция}

\begin{bframe}
\begin{center}
\begin{tikzpicture}
\uncover<1->{
\node[align=left] (n)  at (0,0) {
\from{Если дует ветер, то деревья качаются}\\
\from{Деревья качаются}\\
\conc{Дует ветер}
};
}
\uncover<2>{
\draw[color=accent, ultra thick] (n.north west) -- (n.south east);
\draw[color=accent, ultra thick] (n.north east) -- (n.south west);

};

\end{tikzpicture}
\end{center}
\end{bframe}

\begin{Person}{Pierce}{Чарльз Пирс}{1839--1914}
Пусть наблюдается некий удивительный факт C. Если бы факт A имел место, то C казалось бы вполне нормальным. Это основание предположить, что А на самом деле имел место.
\end{Person}

\begin{bframe}
\begin{Reason}
\from{Наблюдается $B$}
\from{$A$ является причиной $B$}
\from{$A$ {\it наиболее простое объяснение}}
\conc{$A$ имело место}
\end{Reason}
\end{bframe}


\EmptyPictureFrame{Field2}

\begin{frame}\frametitle{Критерии объяснения}

\renewcommand{\ReasonWidth}{6cm}

\begin{columns}
\column{0.35\textwidth}
\uncover<1->{Базовые критерии}

\begin{itemize}
\item<2-> Адекватность
\item<3-> Простота
\end{itemize}

\uncover<4->{Дополнительные критерии}
\begin{itemize}
\item<5-> Глубина
\item<7-> Ширина
\item<14-> Фальсифицируемость
\item<24-> Консервативность
\item<26-> Скромность

\end{itemize}

\column{0.65\textwidth}

\only<6>{
\say{Мама}{Почему ваза разбилась?}
\say{Сын}{Потому что упала}
}


\from[8-13]{Не включается компьютер}
\from[10-13]{Не работает зарядное устройство}
\from[12-13]{В помещении нет света}
\conc[11-12]{И компьютер, и зарядное устройство сломались}
\conc[9-10]{Компьютер сломался}
\conc[13]{Нет электричества}

\say[15-17]{Алиса}{Что-то я неважно себя чувствую}
\say[16-17]{Урсула}{У вас, наверное, аура потускнела}
\say[17]{Урсула}{Давайте я вам проведу чистку, всего  за 100 у.е...}

\say[18-23]{Джон}{Почему нет воды?}
\say[19-23]{Боб}{Рептилоиды опять досаждают! Конечно, это они отключили воду}
\say[20-23]{Джон}{Но я читал в газете, что трубы не ремонтировали уже 30 лет...}
\say[21-23]{Боб}{Это ложь. Просто рептилоиды и газеты контролируют.}
\say[22-23]{Джон}{Надо вывести их на чистую воду! Давай всем расскажем об этом!}
\say[23]{Боб}{Не выйдет. Тебя упекут в сумасшедший дом, ведь врачи -- тоже рептилоиды}

\from[25]{Я пришел в мой любимый ресторан в 18-00, и он не работает}
\conc[25]{Наверное, теперь этот ресторан будет работать только по утрам}

\from[27-]{Трамвая нет уже 20 минут}
\conc[28]{Наверное, трамваи в городе больше никогда ходить не будут}
\conc[29]{Наверное, трамваи на линии больше никогда ходить не будут}
\conc[30]{Наверное, трамваи на линии некоторое время не ходят}
\conc[31]{Наверное, трамваи на линии некоторое время не ходят из-за аварии}


\end{columns}
\end{frame}

\section{Аналогия}

\begin{bframe}
\begin{Reason}
\from{Лось похож на оленя: оба с рогами и копытами}
\from{Олень травоядный}
\conc{Возможно, лось травоядный}
\end{Reason}
\end{bframe}

\begin{bframe}
\begin{Reason}
\from{A схоже с B: они обладают свойствами X, Y, Z}
\from{B обладает свойством W}
\conc{Возможно, A обладает свойством W}
\end{Reason}
\end{bframe}

\begin{bframe}
\begin{Reason}
\from{Меперидин похож на морфин: они имеют схожее строение и вызывают S-образное искривление хвоста у мышей}
\from{Морфин -- анальгетик}
\conc{Возможно, меперидин -- анальгетик}
\end{Reason}
\end{bframe}

\begin{bframe}
\begin{Reason}
\from{Сфера похожа на окружность: это множества точек, равноудаленных от данной}
\from{Окружность имеет наибольшую площадь из фигур одинакового периметра}
\from{Периметр похож на площадь поверхности, а площадь окружности -- на объем сферы}
\conc{Возможно, сфера имеет наибольший объем из фигур одинаковой площади поверхности}
\end{Reason}
\end{bframe}

\begin{bframe}
Беспокойство также помогает решить проблемы, как жевательная резинка -- алгебраический пример
\end{bframe}

\begin{bframe}
\from{Зебра похожа на лося: они травоядные и обладают копытами}
\from{Лось обладает рогами}
\conc{Возможно, у зебры есть рога}
\end{bframe}

\begin{bframe}
\begin{center}
\begin{Reason}
\from{A схоже с B: они обладают свойствами X, Y, Z}
\from{B обладает свойством W}
\conc{A обладает свойством W}
\end{Reason}

\vskip1cm

\uncover<2->{
Это потому, что X, Y, Z являются причинами W.
}
\end{center}
\end{bframe}

\begin{bframe}
\begin{Reason}
\from{Спутник Юпитера Европа похожа на <<черных курильщиков>>: есть вода и источники энергии}
\from{Черные курильщики обитаемы}
\conc{Возможно, Европа обитаема}
\end{Reason}
\end{bframe}

\EmptyPictureFrame{Smoker}


\end{document}