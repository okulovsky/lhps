\documentclass[aspectratio=169]{beamer}

\usepackage{lhps}
\renewcommand{\ImageSource}{L6-Images/}




\begin{document}

\begin{bframe}
\begin{center}
\begin{tikzpicture}[x=1.5cm,y=1.5cm]
\node(n0) at (-2,0) {Дикость};
\node(n1) at (2,0) {Прогресс};
\node(n3) at (0,-2) {Бедность};
\node(n2) at (0,2) {Богатство};
\draw(n0) -- (n1);
\draw(n2) -- (n3);
\uncover<+->{\node at(1,1) {Мы};}
\uncover<.->{\node at(-1,-1) {Другие};}
\uncover<+>{\node at(1,-1) {Мы?};} % мы нищие духом
\uncover<+>{\node at(-1,1) {Мы?};} % мы убиваем ради богатства
\uncover<+>{\node at(-1,1) {Другие?};} % 
\uncover<+>{\node at(1,-1) {Другие?};}
\end{tikzpicture}
\end{center}
\end{bframe}


\begin{bframe}
\begin{center}
\begin{tikzpicture}[x=1.5cm,y=1.5cm]
\node(n0) at (-2,0) {Коммунизм};
\node(n1) at (2,0) {Капитализм};
\node(n3) at (0,-2) {Рабство};
\node(n2) at (0,2) {Свобода};
\draw(n0) -- (n1);
\draw(n2) -- (n3);

\uncover<+->{\node at(1,1) {Мы};}
\uncover<.->{\node at(-1,-1) {Другие};}
\uncover<+>{\node at(1,-1) {Мы?};}
\uncover<+>{\node at(-1,1) {Мы?};}
\uncover<+>{\node at(-1,1) {Другие?};}
\uncover<+>{\node at(1,-1) {Другие?};}
\end{tikzpicture}
\end{center}
\end{bframe}


\begin{Person}{Jefferson}{Томас Джефферсон}{1743--1826}
\Citation{
Все люди созданы равными и наделены Творцом определенными неотъемлемыми правами, к числу которых относится право на жизнь, на свободу и на стремление к счастью
}{Декларация независимости США, 1776}
\end{Person}

\begin{bframe}\frametitle{Борьба за права}
\begin{itemize}
\item<+-> Рабочие
\item<+-> Коренные жители колоний
\item<+-> Женщины
\item<+-> Расовые меньшинства
\item<+-> ЛГБТ
\end{itemize}
\end{bframe}

\begin{bframe}\frametitle{Периодизация феминизма}
\begin{itemize}
\item<+-> Первая волна: конец XIX -- начало XX веков
\item<+-> Вторая волна: 1960 -- 1990
\item<+-> Третья волна: 1990 -- наше время
\end{itemize}
\end{bframe}

\begin{bframe}

\begin{center}
Wer wirdst der n\"achste Bundeskanzler\uncover<1>{?}\uncover<2>{oder Bundeskanzlerin?}
\end{center}
\end{bframe}

\begin{bframe}
\begin{center}
\begin{tikzpicture}[x=1.5cm,y=1.5cm]
\node(n0) at (-2,0) {Женщины};
\node(n1) at (2,0) {Мужчины};
\node(n3) at (0,-2) {Объект};
\node(n2) at (0,2) {Субъект};
\draw(n0) -- (n1);
\draw(n2) -- (n3);

\uncover<+->{\node at(1,1) {Герой};}
\uncover<.->{\node at(-1,-1) {Другой};}
\uncover<+>{\node at(-1,1) {Герой?};}
\uncover<+>{\node at(1,-1) {Герой?};}
\uncover<+>{\node at(-1,1) {Другой?};} 
\uncover<+>{\node at(1,-1) {Другой?};} 
\end{tikzpicture}
\end{center}
\end{bframe}

\begin{frame}\frametitle{Правозащита}

\uncover<+->{Маргинализируемые группы}

\begin{itemize}
\item<+-> Страдающие ожирением
\item<+-> Чайлдфри
\end{itemize}

\uncover<+->{Схема}
\begin{itemize}
\item<+-> Фиксация положения
\item<+-> Апология
\item<+-> Анализ культуры и выделение стереотипов
\item<+-> Указание выгодоприобретателя
\item<+-> Деконструкция стереотипов
\item<+-> Принятие в обществе
\item<+-> Законодательное регулирование
\end{itemize}

\end{frame}


\begin{bframe}\frametitle{Деконструкция науки}


\begin{center}
\begin{tikzpicture}[x=1.5cm,y=1.5cm]
\node(n0) at (-2,0) {Не-наука};
\node(n1) at (2,0) {Наука};
\node(n3) at (0,-2) {Неведенье};
\node(n2) at (0,2) {Познание};
\draw(n0) -- (n1);
\draw(n2) -- (n3);

\uncover<+->{\node at(1,1) {Мы};}
\uncover<.->{\node at(-1,-1) {Другие};}
\uncover<+>{\node at(1,-1) {Мы?};}
\uncover<+>{\node at(-1,1) {Мы?};}
\uncover<+>{\node at(1,-1) {Другие?};}
\uncover<+>{\node at(-1,1) {Другие?};}
\end{tikzpicture}
\end{center}

\end{bframe}


\begin{bframe}\frametitle{Защита лженауки}

\begin{itemize}
\item<+-> Фиксация положения
\item<+-> Апология
\item<+-> Анализ культуры и выделение стереотипов
\item<+-> Указание выгодоприобретателя
\item<+-> Деконструкция стереотипов
\item<+-> Принятие в обществе
\item<+-> Законодательное регулирование
\end{itemize}
\end{bframe}


\begin{Person}{Feyerabend}{Пол Фейерабенд}{1924--1994}

\Citation{

Всё дозволено

}
{
<<Против метода>>, 1975
}

\end{Person}

\end{document}