\documentclass[aspectratio=169]{beamer}
\usepackage{framed}
\usepackage{lhps}

\renewcommand{\ImageSource}{L3-Images/}
\begin{document}


\begin{frame}{Современный научный метод}
\begin{center}
\begin{tikzpicture}[x=4cm,y=-2cm]

\uncover<1->{\node (n1) at (0,0) {Реальность};}
\uncover<3->{\node (n2) at (1,0) {Данные};}
\uncover<2->{\draw[->] (n1) -- node[above]{\it Наблюдения} (n2);}
\uncover<5->{\node (n3) at (2,0) { Гипотеза};}
\uncover<4->{
\draw[->] (n2) -- node[above]{\begin{tabular}{c}\it Индукция\\\it Абдукция\end{tabular}} (n3);
}


\uncover<7->{\node (n4) at (1.5,1) {Граничные условия};}
\uncover<6->{\draw[->] (n3) -- node[right]{\it $\ \ $Дедукция} (n4);}
\uncover<9->{\draw[->] (n4) -- node[left]{\it Эксперимент$\ $} (n2);}
\uncover<11->{\node (n5) at (3,0) { Теория};}
\uncover<10->{\draw[->] (n3) -- node[above]{\it Проверки} (n5);}

\end{tikzpicture}
\end{center}
\end{frame}



\end{document}