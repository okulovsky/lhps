\documentclass[aspectratio=169]{beamer}

\usepackage{lhps}
\renewcommand{\ImageSource}{L6-Images/}




\begin{document}

\begin{bframe}\frametitle{Деконструкция без границ}

\begin{itemize}
\item<+-> Массовость
\item<+-> Отсутствие реконструкции
\item<+-> Тотальность
\end{itemize}

\end{bframe}

\begin{bframe}\frametitle{Основания нарративов}
\begin{itemize}
\item<+-> Законы логики
\item<+-> Законы природы
\item<+-> Анатомия и физиология
\item<+-> Особенности мышления человека как вида
\item<+-> Культурные и исторические
\end{itemize}
\end{bframe}

\EmptyPictureFrame{levitation}

\EmptyPictureFrame{matrix}

\EmptyPictureFrame{UpsideDown}

\EmptyPictureFrame{Versailles}

\EmptyPictureFrame{Englishgarten}

\EmptyPictureFrame{Collage}

\begin{bframe}\frametitle{Высшее образование}

\begin{itemize}
\item Посещение лекций 
\item Получение знаний
\item Выполнение практических заданий
\item Получение профессиональных навыков
\item Сдача экзаменов, подтверждающих знания и навыки
\end{itemize}

\end{bframe}

\begin{bframe}\frametitle{Восприятие науки}
\begin{itemize}
\item<+-> Сложная
\item<+-> Бесполезная
\item<+-> Опасная
\item<+-> Жестокая
\item<+-> {\bf Больше не имеет монополии на истину}
\end{itemize}

\end{bframe}

\begin{Person}{Latynina}{Юлия Латынина}{1966 -- н.в.}

\Citation{
\only<1>{
Мы живем во время светских религий. Что такое светская религия? Обычная религия представляет собой сумму утверждений, не опровергаемых опытом. Например, католики вам скажут, что во время мессы хлеб и вино физически превращаются в плоть и кровь Христову. Но если вы предложите провести химический анализ, вам объяснят, что это происходит способом, превосходящим человеческое понимание. Таким образом, обычная религия заведомо ставит себя вне опыта. 
}

\only<2>{
Светская религия представляет собой сумму утверждений, легко проверяемых опытом[, например,] «проклятые ученые травят нас ГМО». В отличие от тезиса о пресуществлении вина, эти утверждения легко подтвердить или опровергнуть. Тем не менее особенность нашего времени заключается в том, что никто, в них верующий, опытом их не собирается проверять.
}
}
{
<<Проповедь о светских религиях>>, 2014
}


\end{Person}


\begin{bframe}\frametitle{Мироощущение модерна}
\begin{itemize}
\item Ощущение сверхспособностей
\item Сциентизм
\item Либерализм
\item Мессианство
\item Светлое будущее
\end{itemize}
\end{bframe}

\EmptyPictureFrame{demonstration}

\EmptyPictureFrame{elections}

\EmptyPictureFrame{chumak}

\PictureFrame{rand}{Айн Ранд. Атлант расправил плечи}

\begin{bframe}\frametitle{Стоит ли читать?}11
За
\begin{itemize}
\item Очень точное описание диалогов пост-модерна
\end{itemize}
\end{bframe}

\end{document}