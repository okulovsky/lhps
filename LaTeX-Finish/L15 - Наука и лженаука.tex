\documentclass[aspectratio=169]{beamer}

\usepackage{lhps}
\renewcommand{\ImageSource}{L5-Images/}

\begin{document}

\begin{frame}{Демаркация}
\begin{center}
\begin{tikzpicture}[y=-1.15cm, x=7cm]
\uncover<+->{
\node[draw=black,align=center] (n0) at (0,-0.2) {Консервативное, о реальности,\\ соответствует этике и методу?};
\node (m0) at (1,-0.2) {Нормальная наука};
\draw[->](n0) -- node[above]{\it да} (m0);
}

\uncover<+->{
\node[draw=black] (n1) at (0,1) {О реальности?};
\draw[->] (n0) -- node[left]{\it нет} (n1);
}
\uncover<+->{
\node (m1) at (1,1) {Не наука};
\draw[->](n1) -- node[above]{\it нет} (m1);
}

\uncover<+->{
\node[draw=black] (n2) at (0,2) {О чужой работе?};
\draw[->] (n1) -- node[left]{\it да}  (n2);
}

\uncover<+->{
\node[align=center] (m2) at (1,2) {Научпоп};
\draw[->](n2) -- node[above]{\it да} (m2);
}


\uncover<+->{
\node[draw=black] (n3) at (0,3) {Не консервативное, но остальное ОК?};
\draw[->] (n2) -- node[left]{\it нет}  (n3);
}

\uncover<+->{
\node (m3) at (1,3) {Пограничная наука};
\draw[->](n3) -- node[above]{\it да} (m3);
}

\uncover<+->{
\node[draw=black] (n4) at (0,4) {Противоречит науке?};
\draw[->] (n3) -- node[left]{\it нет}  (n4);
}
\uncover<+->{
\node (m4) at (1,4) {Паранаука};
\draw[->](n4) -- node[above]{\it нет} (m4);
}

\uncover<+->{
\node (n5) at (1,5) {Лженаука};
\draw (n4) -- (0,5);
\draw[->] (0,5) -- node[above]{\it да} (n5);
}
\end{tikzpicture}
\end{center}
\end{frame}

\EmptyPictureFrame{Tarot}

\begin{Person}{Hahnemann}{Самуэль Ганеманн}{1755--1843}
То, что вызывает симптомы у здорового человека, может в меньших дозах вылечить те же симптомы у больного
\end{Person}

\begin{bframe}\frametitle{Признаки лженауки}
\item Глобальность проблемы
\item Простые по сути решения
\item Ad nauseam
\item Обращение к широкой публике
\item Теория заговора
\item Имитация процедур института науки
\end{bframe}



\end{document}