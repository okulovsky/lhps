\documentclass[aspectratio=169]{beamer}
\usepackage{framed}
\usepackage{lhps}
\usepackage{changepage}
\usetikzlibrary{calc}
\renewcommand{\ImageSource}{L2-Images/}

\definecolor{rosso}{RGB}{220,57,18}
\definecolor{giallo}{RGB}{255,153,0}
\definecolor{blu}{RGB}{102,140,217}
\definecolor{verde}{RGB}{16,150,24}
\definecolor{viola}{RGB}{153,0,153}

\makeatletter

\tikzstyle{chart}=[
    legend label/.style={font={\scriptsize},anchor=west,align=left},
    legend box/.style={rectangle, draw, minimum size=5pt},
    axis/.style={black,semithick,->},
    axis label/.style={anchor=east,font={\tiny}},
]

\tikzstyle{bar chart}=[
    chart,
    bar width/.code={
        \pgfmathparse{##1/2}
        \global\let\bar@w\pgfmathresult
    },
    bar/.style={very thick, draw=white},
    bar label/.style={font={\bf\small},anchor=north},
    bar value/.style={font={\footnotesize}},
    bar width=.75,
]

\tikzstyle{pie chart}=[
    chart,
    slice/.style={line cap=round, line join=round, very thick,draw=white},
    pie title/.style={font={\bf}},
    slice type/.style 2 args={
        ##1/.style={fill=##2},
        values of ##1/.style={}
    }
]

\pgfdeclarelayer{background}
\pgfdeclarelayer{foreground}
\pgfsetlayers{background,main,foreground}


\newcommand{\pie}[3][]{
    \begin{scope}[#1]
    \pgfmathsetmacro{\curA}{90}
    \pgfmathsetmacro{\r}{1}
    \def\c{(0,0)}
    \foreach \v/\s in{#3}{
        \pgfmathsetmacro{\deltaA}{\v/100*360}
        \pgfmathsetmacro{\nextA}{\curA + \deltaA}
        \pgfmathsetmacro{\midA}{(\curA+\nextA)/2}

        \path[slice,\s] \c
            -- +(\curA:\r)
            arc (\curA:\nextA:\r)
            -- cycle;
        \pgfmathsetmacro{\d}{max((\deltaA * -(.5/50) + 1) , .5)}

        \begin{pgfonlayer}{foreground}
        \path \c -- node[pos=\d,pie values,values of \s]{} +(\midA:\r);
        \end{pgfonlayer}

        \global\let\curA\nextA
    }
    \end{scope}
}

\begin{document}



\section{Индукция}

\begin{bframe}
\Citation{
Если попадется тебе на дороге птичье гнездо на каком-либо дереве или на земле, с птенцами или яйцами, и мать сидит на птенцах или на яйцах, то не бери матери вместе с детьми:
мать пусти, а детей возьми себе, чтобы тебе было хорошо, и чтобы продлились дни твои.
}{Второзаконие, Глава 22, Притч. 10 12}
\end{bframe}

\begin{bframe}
\Citation{И был вечер, и было утро: день четвёртый. И сказал Бог: да произведёт вода пресмыкающихся, душу живую; и птицы да полетят над землёю, по тверди небесной. И сотворил Бог рыб больших и всякую душу животных пресмыкающихся, которых произвела вода, по роду их, и всякую птицу пернатую по роду её. И увидел Бог, что [это] хорошо. И благословил их Бог, говоря: плодитесь и размножайтесь, и наполняйте воды в морях, и птицы да размножаются на земле
}
{Бытие, Глава 1}
\end{bframe}

\begin{Person}{Descartes}{Рене Декарт}{1596--1650}
\Citation{
\only<+>{
Так как мы рождаемся детьми и составляем разные суждения о вещах прежде, чем достигнем полного употребления своего разума, то многие предрассудки отклоняют нас от познания истины; избавиться от них мы, по-видимому, можем не иначе, как постаравшись раз в жизни усомниться во всём том, в чём найдём хотя бы малейшее подозрение недостоверности.}

\only<+->{ Если мы станем отвергать всё то, в чём каким бы то ни было образом можем сомневаться, и даже будем считать всё это ложным, то хотя мы легко предположим, что нет никакого Бога, никакого неба, никаких тел и что у нас самих нет ни рук, ни ног, ни вообще тела, однако же не предположим также и того, что мы сами, думающие об этом, не существуем: ибо нелепо признавать то, что мыслит, в то самое время, когда оно мыслит, не существующим. Вследствие чего это познание: я мыслю, следовательно существую, — есть первое и вернейшее из всех познаний}
}
{<<Первоначала философии>>, 1644}
\end{Person}

\begin{Person}{Bacon}{Фрэнсис Бэкон}{1561--1626}
\Citation{
Эмпирики,  подобно  муравью, только  собирают  и  довольствуются  собранным.
Рационалисты,  подобно  паукам,  производят ткань  из самих  себя. Пчела  же
избирает средний способ: она извлекает материал из садовых и полевых цветов,
но располагает и изменяет  его  по своему умению.  Не отличается от этого  и
подлинное дело философии.
}{
Новый Органон, 1620 г.
}
\end{Person}

\begin{bframe}
\begin{Reason}
	\from{Все люди смертны}
	\from{Сократ -- человек}
	\conc{Сократ -- смертен}
\end{Reason}
\end{bframe}

\begin{Person}{Euklid}{Евклид}{ок. 325 до н.э. -- ок. 265 до н.э.}
\Citation{
\begin{footnotesize}
\begin{enumerate}
\item От всякой точки до всякой точки можно провести прямую.
\item  Ограниченную прямую можно непрерывно продолжать по прямой.
\item  Из всякого центра всяким раствором может быть описан круг.
\item  Все прямые углы равны между собой.
\item  Если прямая, пересекающая две прямые, образует внутренние односторонние углы, меньшие двух прямых, то, продолженные неограниченно, эти две прямые встретятся с той стороны, где углы меньше двух прямых.
\end{enumerate}
\end{footnotesize}
}{Начала, ок. 300 до н.э.}
\end{Person}

\begin{Person}{Piphagor}{Пифагор}{571 до н.э. -- 495 до н.э.}
Квадрат гипотенузы равен сумме квадратов катетов
\end{Person}

\section{Простое обобщение}

\begin{bframe}
\begin{Reason}
	\from{Все наблюдаемые на текущий момент вороны черные}
	\conc{Все вороны черные}
\end{Reason}
\end{bframe}

\begin{bframe}
\begin{Reason}
\from{Все наблюдаемые на текущий момент $X$ есть $A$}
\conc{Все $X$ есть $A$}
\end{Reason}
\end{bframe}

\begin{bframe}
\begin{Reason}
\from[1-]{Все наблюдаемые на текущий момент породы дерева плавают}
\conc[2]{Все породы дерева плавают}
\conc[3]{{\it Почти} все породы дерева плавают}
\end{Reason}
\end{bframe}


\section{Ошибки, связанные с обобщением}

\begin{bframe}
\begin{Reason}
\from[1-]{В Великобритании отказались от абсолютной монархии, и добились научно-технического прогресса}
\from[2-]{Во Франции свергли абсолютную монархию, и добились научно-технического прогресса}
\from[3-]{В США никогда не было абсолютной монархии и добились научно-технического прогресса}
\conc[4-]{Научно-технический прогресс возможен там, где нет абсолютной монархии}
\end{Reason}
\uncover<5->{\begin{center}Да нет же, это просто совпадение!\end{center}}
\end{bframe}


\begin{bframe}
\uncover<+->{Опять дожливые выходные! Уже вторую неделю! Все! Лето кончилось!}

\uncover<+->{
\begin{Reason}
\from{Эти выходные были дождливыми}
\from{Предыдущие выходные были дождливыми}
\conc{Выходные {\it и дальше будут} дождливы}
\end{Reason}
}
\end{bframe}

\begin{bframe}
\begin{Reason}
\say[+-]{Джон}{Я собираюсь купить мобильный телефон. Думаю, остановлюсь на чем-нибудь фирмы Lark...}
\say[+-]{Джейн}{Только не Lark! Боб купил такой, а телефон потерял несколько SMS. В результате Боб пропустил собеседование, не смог устроится в престижную фирму и теперь вынужден продавать мороженое с тележки!}
\say[+-]{Джон}{Оу... Да, наверное, Lark --- не лучший выбор...}
\end{Reason}
\end{bframe}

\begin{bframe}
\begin{Reason}
\from[1]{Философия -- чисто немецкое занятие}
\from[2-]{Кант, Маркс, Гегель, Фейербах -- все они немцы}
\conc[3-]{Философия -- чисто немецкое занятие}
\end{Reason}
\end{bframe}



\PictureFrame{LiterallyDigest}{Президентские выборы в США 1936 года}{Выпуск Literally Digest с результатами опроса}


\begin{bframe}
\begin{Reason}
\say[+-]{Джон}{Курение вредит здоровью. Многочисленные исследования показывают, что...}
\say[+-]{Джейн}{Какой вздор! Уинстон Черчилль курил как паровоз, но отметил 90-летие, и до самой смерти был здоров и полон сил! }
\end{Reason}
\end{bframe}

\begin{bframe}\frametitle{Статистическое обобщение}
\begin{Reason}
\from[1-]{N\% от всех A в выборке обладают свойством B}
\conc[2-]{N\% всех A обладают свойством B}
\conc[3]{Пусть X есть A. Тогда с вероятностью N\% он обладает свойством B}
\end{Reason}
\end{bframe}


\section{Простые проблемы со статистикой}

%https://ru.wikipedia.org/wiki/Как_лгать_при_помощи_статистики

\EmptyPictureFrame{LieWithStatistics}




\begin{bframe}
\uncover<+->{Джон заканчивает работать между 18-00 и 19-00 в абсолютно случайный момент. Он спускается в метро и садится в первый подошедший поезд. При этом поезда одного направлении везут его к родителям, а поезда другого направления -- к невесте. Мать Джона жалуется, что сын редко у них бывает. Права ли мать?}

\uncover<+->{
\begin{center}
\begin{tabular}{l l l l l l l}
К родителям & 18-00 & 18-10 & 18-20 & 18-30 & 18-40 & 18-50 \\
К невесте & 18-09 & 18-19 & 18-29 & 18-39 & 18-49 & 18-59\\
\end{tabular}
\end{center}}

\uncover<+->{
\begin{center}
\begin{tikzpicture}[x=2cm,y=0.5cm]
\foreach \x in {0,1,2,3,4,5}
{
	\draw[chartbase] (\x,0) rectangle(\x+1,1);
	\draw[fill=chartdarkaccent] (\x+0.8,0) rectangle(\x+1,1);
}
\end{tikzpicture}
\end{center}
}

\end{bframe}

\newcommand{\rad}{0.3}
\begin{bframe}
\begin{center}
\begin{tikzpicture}[y=-1cm]
\uncover<1->{\draw (0,0) circle(\rad);
\draw[chartbase,fill=chartdarkaccent] (1,0) circle(\rad);
\draw[chartbase] (2,0) circle(\rad);
\draw[chartbase] (3,0) circle(\rad);
\draw[chartbase,fill=chartdarkaccent] (4,0) circle(\rad);
\draw[chartbase,fill=chartdarkaccent] (5,0) circle(\rad);

\draw[chartbase] (0,1) circle(\rad);
\draw[chartbase] (1,1) circle(\rad);
\draw[chartbase] (2,1) circle(\rad);
\draw[chartbase,fill=chartdarkaccent] (3,1) circle(\rad);
\draw[chartbase] (4,1) circle(\rad);
\draw[chartbase] (5,1) circle(\rad);}

\only<2>{
\draw[chartbase] (0,2.5) circle(\rad);
\draw[chartbase] (1,2.5) circle(\rad);
\draw[chartbase,fill=chartdarkaccent](2,2.5) circle(\rad);
\draw[chartbase] (3,2.5) circle(\rad);
\draw[chartbase] (4,2.5) circle(\rad);
\draw[chartbase,fill=chartdarkaccent] (5,2.5) circle(\rad);
}

\uncover<3->{
\draw[chartbase] (0,2.5) circle(\rad);
\draw[chartbase] (1,2.5) circle(\rad);
\draw[chartbase] (2,2.5) circle(\rad);
\draw[chartbase] (3,2.5) circle(\rad);
\draw[chartbase] (4,2.5) circle(\rad);
\draw[chartbase] (5,2.5) circle(\rad);
}

\end{tikzpicture}
\end{center}

\uncover<4->{
$$
S_1=6\cdot \frac{2}{3} = 4
$$

$$
S_2 = 6\cdot \left( \frac{2}{3}\cdot\frac{2}{3} + \frac{1}{3}\cdot\frac{1}{3} - \frac{2}{3}\cdot\frac{2}{3}\cdot\frac{1}{3}\cdot\frac{1}{3}\right)=6 \cdot \frac{41}{81} \approx 3.03 < 4
$$
}
\end{bframe}

\EmptyPictureFrame{Lacrimosa}

\begin{bframe}
\begin{adjustwidth}{1.5em}{}
\begin{tikzpicture}
[
    pie chart,
    slice type={comet}{chartdarkaccent},
    slice type={legno}{chartlightaccent},
    pie values/.style={font={\small}}
]

\only<1>{
    \pie[scale=2]{Христиане}{99/comet,1/legno}
  \pie[scale=2,shift={(4cm,0cm)}]{Сатанисты}{60/comet,40/legno}
	\node at (8cm,-3cm) {Сатанисты};
	\node at (0,-3cm) {Христиане};
  
   
}

\only<2>{
    \pie[scale=3]{Христиане}{99/comet,1/legno}
    \pie[scale=0.2,shift={(35cm,0cm)}]{Сатанисты}{60/comet,40/legno}
	\node at (7cm,-1.5cm) {Сатанисты ($\approx 100 000$)};
	\node at (0,-3.5cm) {Христиане ($\approx 2 000 000 000$)};
}

\end{tikzpicture}
\end{adjustwidth}
\end{bframe}



\end{document}