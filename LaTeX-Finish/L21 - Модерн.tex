\documentclass[aspectratio=169]{beamer}

\usepackage{lhps}
\renewcommand{\ImageSource}{L6-Images/}




\begin{document}

\EmptyPictureFrame{GameOfThrones}

\begin{bframe}\frametitle{Население крупных европейских городов, тыс. ч.}
\begin{center}
\begin{tabular}{l| r r r r r}
\hline \hline
Город      & 1000         & 1250        & 1500          & 1750             & 2000           \\
\hline \hline
Лондон     & 10           & 100          & 100           & 700              & 14 000         \\
Париж      & 50           & 160          & 185           & 556              & 12 000         \\
Берлин     &              &              & 6             & 90               & 5 000          \\
Рим        &              & 20           &               & 150              & 4 000          \\
\hline \hline 
\end{tabular}\\[10pt]

\end{center}
\end{bframe}


\PictureFrame{WorldEconomicHistory}{История мировой экономики в одном графике}{Среднедушевой доход по времени с X в. до н.э. по настоящее время в относительных единицах}

\PictureFrame{Zelikov}{Филипп Зеликов, The Modern World}{
https://www.coursera.org/learn/modern-world}

\EmptyPictureFrame{Columbus}


\begin{bframe}\frametitle{Классическая картина мира}
\begin{itemize}
\item<+-> Скепсис к авторитетам и традициям
\item<+-> Вера в упорядоченность и предсказуемость мира
\item<+-> Бог-часовщик
\item<+-> Вера в разум и в человека
\end{itemize} 
\end{bframe}


\begin{Person}{Hobbes}{Томас Гоббс}{1588--1679}

\Citation{
\begin{itemize}
\item Естественное состояние человека vs государство
\item Общественный договор: добровольный отказ от свободы в обмен на безопасность
\end{itemize}
}{
Левиафан, 1651
}
\end{Person}

\begin{Person}{Locke}{Джон Локк}{1632--1704}
\begin{itemize}
\item Идея личной свободы и естественного права
\item Право на смену власти, которая не исполняет общественный договор и ограничивает естественное право 
\end{itemize}
\end{Person}

\begin{Person}{Kant}{Иммануил Кант}{1724--1804}
Категорический императив:
\begin{itemize}
\item поступай так, чтобы правило твоей воли могло иметь силу принципа всеобщего законодательства; такое правило должно распространятся на всех, в том числе и на тебя;
\item относится к другим людям надо также, какого отношения ты ждешь к своей персоне;
\item к человеку нельзя относится как к средству для решения своих интересов.
\end{itemize}
\end{Person}

\begin{Person}{Smith}{Адам Смит}{1723--1790}

\begin{itemize}
\item Экономический человек
\item Невидимая рука рынка
\end{itemize}

\end{Person}

\begin{Person}{Hegel}{Георг Вильгельм Фридрих Гегель}{1770--1831}

\begin{itemize}
\item История разумна и является проявлением Мирового Духа
\item Свобода есть важнейший атрибут Духа
\end{itemize}
\end{Person}

\begin{bframe}\frametitle{Промышленная революция}
\begin{itemize}
\item<1-> Машины: ткацкий станок, насос\uncover<2->{, станки для производства станков}
\item<3-> Источники энергии: паровая машина\uncover<4->{, двигатель внутреннего сгоряния, электродвигатель}
\item<5-> Материалы: чугун, сталь, резина, медь
\item<6-> Ресурсы: каменноугольный кокс, руды, химикаты
\item<7-> Транспорт: паровоз, пароход\uncover<8->{, автомобиль, самолет}
\item<9-> Связь: телеграф, телефон, радио
\end{itemize}
\end{bframe}

\begin{bframe}\frametitle{Влияние на общество}
\begin{itemize}
\item<+-> Рост уровня жизни
\item<+-> Развитие науки, рост уровня образования, грамотности
\item<+-> Уменьшение мира
\item<+-> Глобализация
\item<+-> Сильное государство
\item<+-> Социальное расслоение
\item<+-> Ощущение нестабильности
\end{itemize}
\end{bframe}

\begin{bframe}\frametitle{Мироощущение модерна}
\begin{itemize}
\item<+-> Ощущение сверхспособностей
\item<+-> Сциентизм
\item<+-> Либерализм
\item<+-> Гуманизм
\item<+-> Мессианство
\item<+-> Светлое будущее
\end{itemize}
\end{bframe}
\end{document}