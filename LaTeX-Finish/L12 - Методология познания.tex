\documentclass[aspectratio=169]{beamer}
\usepackage{framed}
\usepackage{lhps}

\renewcommand{\ImageSource}{L4-Images/}
\begin{document}

\begin{Person}{Hume}{Дэвид Юм}{1711--1776}
\Citation{
Мы не способны, исходя из одного примера, открыть какую-либо силу связывающую действие с причиной и делающее первое неизменным следствием второй. Мы находим только, что действие в самом деле, фактически, следует за причиной. Толчок, производимый одним бильярдным шаром, сопровождается движением второго. И это все, что является внешним чувствам. Ни в каком единичном, частном случае причинности нет ничего такого, что могло бы вызвать идею силы, или необходимой связи.
}{<<Исследование о человеческом разумении>>, 1748}

\end{Person}

\begin{bframe}
\begin{center}
\begin{tabular}{c c c c}
&& \uncover<3->{Аналитические} 	
& \uncover<3->{Синтетические} \\

& & \\[-0.2cm] 

  \uncover<2->{Априорные} 		
&& \uncover<4->{Силлогизм}
& \uncover<7->{Закон Ньютона}\uncover<8>{?} \\

  \uncover<2->{Апостериорные}
&& \uncover<6->{---}
& \uncover<5->{Эмпирический факт} \\
\end{tabular}
\end{center}
\end{bframe}


\begin{Person}{Kant}{Иммануил Кант}{1724--1804}

\Citation{}

{<<Критика чистого разума>>, 1781}

\end{Person}


\end{document}