\documentclass[aspectratio=169]{beamer}
\usepackage{framed}
\usepackage{lhps}

\renewcommand{\ImageSource}{L3-Images/}
\begin{document}


\begin{Timeline}{1500}{1900}
\tlbio{1543}{0}{Гелиоцентризм}{Kopernik}{Николай Коперник}{1473--1543}{предложил гелиоцентризм}
\tlbio{1609}{0}{Законы Кеплера}{Kepler}{Иоганн Кеплер}{1571--1630}{описал законы движения планет}
\tlbio{1687}{0.1}{Законы Ньютона}{Newton}{Исаак Ньютон}{1643 -- 1727}{создал классическую механику}

\tlbio{1597}{-0.1}{Термометр}{Galilei}{Галилео Галилей}{1564--1642}{кроме прочего, изобрел термометр}
\tlbio{1650}{-0.1}{Вакуумный насос}{Guericke}{Отто фон Герике}{1602--1686}{изобрел вакуумный насос}
\tlbio{1824}{-0.15}{Работы Карно}{Carnot}{Сади Карно}{1796--1832}{предложил первое начало термодинамики}
\tlsideitem{1870}{0.55}{Начала термодинамики}


\tlbio{1747}{-0.3}{Природа молнии}{Franklin}{Бенджамин Франклин}{1706--1790}{установил электрическую природу молнии}
\tlbio{1791}{-0.3}{Электричество в мышцах}{Galvani}{Луиджи Гальвани}{1737--1798}{установил, что электричество сокращает мышцы}
\tlbio{1800}{-0.25}{Гальванический элемент}{Volta}{Алессандро Вольта}{1745--1827}{создал батарею постоянного тока}
\tlsideitem{1830}{0}{Исследования электричества}
\tlbio{1873}{0.7}{Законы Максвелла}{Maxwell}{Джеймс Клерк Максвелл}{1831--1879}{вывел законы, описывающие электричество и магнетизм}


\tlbio{1676}{0}{Микроорганизмы}{Levenguk}{Антонио ван Левенгук}{1632--1723}{открыл множество микроорганизмов}
\tlbio{1865}{0.12}{Работы Пастера}{Paster}{Луи Пастер}{1822--1895}{опроверг самозарождение жизни, изобрел вакцинацию}
\tlbio{1859}{-0.27}{Эволюция}{Darwin}{Чарльз Дарвин}{1809--1882}{предложил теорию эволюции}
\tlbio{1863}{-0.1}{Законы Менделя}{Mendel}{Грегор Иоанн Мендель}{1822--1884}{открыл законы наследственности}



\tlbio{1661}{-0.1}{Атомы и элементы}{Boyle}{Роберт Бойль}{1627--1691}{переоткрыл концепции элементов и атомов}
\tlbio{1777}{-0.35}{Теория горения}{Lavoisier}{Антуан Лоран Лавуазье}{1743--1794}{создал килородную теорию горения}
\tlsideitem{1820}{-0.35}{Количественные законы химии}
\tlbio{1869}{0.3}{Периодический закон}{Mendeleev}{Дмитрий Менделеев}{1834--1907}{создал периодическую таблицу элементов}
\end{Timeline}



\end{document}